%----README----
% This TeX file was automatically generated by GNU Emacs's Org Mode.
% For better documentation refer to the corresponding Org file, 
% which you can find in the sources specified bellow.
% Do keep in mind that this file undergoes little to none manual revision.
%----LICENSE---
% Copyright 2021 Jhonny Lanzuisi (jalb97@gmail.com)
% More source files at github.com/JLanzuisi
%
% This program is free software: you can redistribute it and/or modify
% it under the terms of the GNU General Public License as published by
% the Free Software Foundation, either version 3 of the License, or
% (at your option) any later version.
%
% This program is distributed in the hope that it will be useful,
% but WITHOUT ANY WARRANTY; without even the implied warranty of
% MERCHANTABILITY or FITNESS FOR A PARTICULAR PURPOSE.  See the
% GNU General Public License for more details.
%
% You should have received a copy of the GNU General Public License
% along with this program.  If not, see <https://www.gnu.org/licenses/>.
%--------------

\hfuzz1pc
\overfullrule=2cm

\usepackage[spanish,es-noindentfirst]{babel}
\usepackage{csquotes}

\usepackage
[
	includehead,
	includefoot,
	top=.5cm,
	bottom=.5cm,
	left=1.7cm,
	right=7.3cm,
	marginparsep=0.5cm,
	marginparwidth=7cm,
]
{geometry}

\usepackage{mathtools}
\DeclareMathOperator{\Rea}{Re}
\DeclareMathOperator{\Ima}{Im}
\DeclareMathOperator{\car}{car}
\DeclareMathOperator{\traz}{tr}
\DeclareMathOperator{\gen}{gen}
\DeclareMathOperator{\mcm}{mcm}

\usepackage{unicode-math}

\setmainfont{New Computer Modern Book}
\defaultfontfeatures{Scale=MatchLowercase}
\setsansfont{Public Sans}
\setmonofont{Average Mono}
\newcommand{\headfont}{\sffamily}

\frenchspacing
\linespread{1.05}

\setmathfont{New Computer Modern Math}

\usepackage[final]{microtype}

\PassOptionsToPackage{final}{graphicx}
\PassOptionsToPackage{dvipsnames}{xcolor}

\usepackage
{
	xcolor,
	graphicx,
	cancel,
	booktabs,
	hyphenat,
	authoraftertitle,
	pdfpages,
	metalogo
}

\usepackage
[
backend=biber,
backref=true,
citestyle=authoryear-comp,
style=chicago-authordate ,
sorting=ynt
]
{biblatex}
\addbibresource{/home/jhonny/git/Misc-LaTeX-files/bib/general.bib}

\usepackage{url} 
\usepackage{hyperref} 
\definecolor{Carmine}{HTML}{960018}
\newcommand{\linkcolor}{Carmine}
\hypersetup
{
colorlinks=true,
linkcolor=\linkcolor,
urlcolor=\linkcolor,
citecolor=\linkcolor
}
\usepackage[spanish,nameinlink]{cleveref}

\newcommand{\marfont}{}

%\renewcommand*{\marginfont}{\marfont}
\let\oldmarginpar\marginpar
\renewcommand
	{\marginpar}
	[1]
	{
	\oldmarginpar{\raggedright\marfont #1}
	}

\newcounter{nota}
\newcommand
	{\nota}
	[1]
	{
	\refstepcounter{nota}\textsuperscript{\thenota}
	\marginpar{
		\raggedright\itshape\thenota. #1
		}
	}

\renewcommand{\footnote}{\nota}

\usepackage{enumitem} 
\setlist[enumerate]{left=-11pt,nosep}
\setlist[description]{font=\normalfont,leftmargin=\parindent}
\setlist[itemize]{label={\small\textbullet},left=-11pt}

\newcommand{\Asignatura}{}
\newcommand
	{\asignatura}
	[1]
	{\renewcommand{\Asignatura}{#1}}

\makeatletter
\def\@maketitle
	{
	\newpage
	\null
	\let \footnote \thanks
  	\begin{flushleft}\sffamily
  	\marginpar
		{
		\vspace*{.2em}
		\begin{minipage}{.7\marginparwidth}
		\begin{center}
		\includegraphics
			[width=.35\marginparwidth]
			{/home/jhonny/git/LaTeX-University/usb-logo.png}\\
		Universidad Simón Bolívar\\
		Caracas, Venezuela\\
		\end{center}
		\end{minipage}
		}
	{
	\Large\headfont
		\MyTitle
	\par
	}
	\smallskip
	\Asignatura\par
	\smallskip
	\@author,\ \today.
	\end{flushleft}
	\vskip 1.5\baselineskip
	}
\makeatother

\usepackage[final]{listings} 
\lstset
{
numbers=left, numberstyle=\tiny\ttfamily, stepnumber=2, numbersep=5pt, 
basicstyle=\ttfamily, 
stringstyle=\ttfamily,
commentstyle=\itshape,
breaklines=true,
postbreak=\mbox{$\hookrightarrow$\enspace},
columns=flexible
}

\usepackage{caption} 
\captionsetup
{
font={rm},
justification=raggedright,
singlelinecheck=false,
skip=3pt
}

\usepackage[explicit]{titlesec}

\titleformat{\section}[hang]
	{\flushleft\headfont}
	{\addfontfeatures{Numbers=Lining}\thesection}
	{1em}
	{\addfontfeatures{LetterSpace=5}\MakeUppercase{#1}}
	[]
\titlespacing*{\section}
	{0em}
	{1.5\baselineskip}
	{.5\baselineskip}
\titleformat{\subsection}
	{\flushleft\sffamily}
	{\addfontfeatures{Numbers=Lining}\thesubsection}
	{.5em}
	{#1}
	[]
\titlespacing*{\subsection}
	{0em}
	{1\baselineskip}
	{1\baselineskip}
\titleformat{\paragraph}[runin]
	{\scshape}
	{}
	{0em}
	{\MakeLowercase{#1}}
	[.]
\titlespacing*{\paragraph}
	{0em}
	{1\baselineskip}
	{5pt}

\let\oldtoc\tableofcontents
\renewcommand
{\tableofcontents}
{\marginpar{\bigskip\oldtoc}}
\usepackage{titletoc}

\titlecontents{section}
	[0em]
	{\vspace{0pt}}
	{\contentsmargin{0pt}}
	{\contentsmargin{0pt}}
	{\contentspage}
	[\vspace{.3em}]
\titlecontents{subsection}
	[1.5em]                              
	{\vspace{0pt}}
	{\contentsmargin{0pt}\small}
	{\contentsmargin{0pt}\small}        
	{\small\contentspage}                 
	[\vspace{.3em}]

\usepackage{fancyhdr}

\renewcommand{\headrulewidth}{0pt}
\setlength{\headheight}{14pt}

\pagestyle{fancy}
\fancyhf{}
\fancyhead
	[L]
	{
	\ifodd\value{page}\MyTitle\else\Asignatura\fi
	}
\fancyhead[R]{\thepage}
\fancypagestyle
	{plain}
	{
	\fancyhead[R]{}
	\fancyhead[L]{}
	\fancyfoot[R]{}%
	\fancyfoot[L]{}
	\fancyfoot[C]{}
	}

\usepackage[thmmarks]{ntheorem}
  % \usepackage[thmmarks]{ntheorem}
  % 	\theoremstyle{plain}
  % 	\theoremindent0cm
  % 	\theorempreskip{0cm}
  % 	\theorempostskip{0cm}
  % 	\theoremheaderfont{\hspace*{\parindent}\upshape}
  % 	\theorembodyfont{\itshape}
  % 	\theoremseparator{.}
  % 	\newtheorem{teo}{Teorema}[section]
  % 	\newtheorem{cor}{Corolario}[teo]
  % 	\newtheorem{prop}{Proposición}[section]
  % 	\newtheorem{lem}{Lema}[section]
  % 	\theoremstyle{nonumberplain}
  % 	\theoremheaderfont{\normalfont}
  % 	\theorembodyfont{\upshape}
  % 	\newtheorem{proof}{Demostración}
  % 	\theoremstyle{plain}
  % 	\theorempreskip{1em}
  % 	\theorempostskip{1em}
  % 	\theoremheaderfont{\upshape}
  % 	\theorempostwork{\noindent}
  % 	\newtheorem{definition}{Definición}[section]

\theoremstyle{plain}
\theorempreskip{\medskipamount}
\theorempostskip{\medskipamount}
\theorembodyfont{\upshape}
\theoremseparator{.}
{
\theoremheaderfont{\itshape}
\newtheorem{definition}{Definición}
}
{
\theoremheaderfont{\scshape}
\newtheorem{theorem}{Teorema}
}
