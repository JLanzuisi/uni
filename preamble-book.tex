% \usepackage[spanish]{babel}
\usepackage{polyglossia}
\setmainlanguage[spanishoperators=all]{spanish}
\PolyglossiaSetup{spanish}{indentfirst=false}

\usepackage{amsmath}

\DeclareMathOperator{\Rea}{Re}
\DeclareMathOperator{\Ima}{Im}
\DeclareMathOperator{\car}{car}
\DeclareMathOperator{\traz}{tr}
\DeclareMathOperator{\gen}{gen}
\DeclareMathOperator{\mcm}{mcm}


\usepackage{unicode-math}
\setmainfont[Numbers={OldStyle,Proportional},SmallCapsFeatures{LetterSpace=10}]{Libertinus Serif}
\defaultfontfeatures{Scale=MatchLowercase}
\setsansfont{Latin Modern Sans}
\setmonofont{Luxi Mono}
\setmathfont{Libertinus Math}
\newfontfamily{\dispfont}{Libertinus Serif Display}
\newfontfamily{\commentfont}{etbb Italic}

\linespread{1.04}
\frenchspacing
\hfuzz1pc

\usepackage{microtype}

\usepackage{%
	xcolor,%
	marginnote,%
	graphicx,%
	cancel,%
	%tikz,%
	pdfpages,
	imakeidx,
	hyperref,
	epigraph,
	enumitem,
	metalogo,
	cleveref,
	caption,
	%floatrow,
	booktabs,
	%longtable
	csquotes,
}

\usepackage[
backref,
backend=biber,
style=trad-abbrv,
sorting=ynt
]{biblatex}
\addbibresource{../../../bib/general.bib}

\hypersetup{colorlinks=true,allcolors=black}
% Color definitions
\definecolor{darkred}{HTML}{66000c}
\definecolor{darkblue}{HTML}{142850}

\makeindex[columns=3, title=Índice Alfabético, intoc]

\setlist[enumerate]{label=\arabic*,left=-11pt}
\setlist[description]{font=\normalfont,leftmargin=\parindent,nosep}
\setlist[itemize]{label={\footnotesize\textbullet},left=-11pt}

%\floatsetup{footposition=bottom}% For all floats

%\floatsetup[longtable]{LTcapwidth=table}% https://tex.stackexchange.com/a/345772/120853

%\floatsetup[table]%
%{%
	%style=plaintop,% Always above, no matter where \caption is called
	%footnoterule=none,%
	%footskip=.35\skip\footins,%
%}%

%\floatsetup[figure]%
%{%
	%capbesidewidth=figure,
	%capbesideposition=right,%
	%capbesidesep=quad,%
%}%

\captionsetup{font={small},indention=.5cm,justification=raggedright}

\usepackage[explicit]{titlesec}
\titleformat{\part}[display]
	{\flushleft\fontsize{22}{22}\selectfont}
	{\LARGE Parte\;\scshape\MakeLowercase{\thepart}}
	{3em}
	{#1}
	[]
\titleformat{\chapter}[hang]
{\flushleft\itshape\fontsize{14}{14}\selectfont}
	{\hspace{-1.2em}\thechapter\;\rule[-5.5pt]{1pt}{23pt}}
	{.3em}
	{#1}
	[]
\titlespacing*{\chapter}
	{0em}
	{0em}
	{1\baselineskip}
\titleformat{\section}[runin]
	{\normalfont}
	{\hspace*{-2.25em}\thesection}
	{5pt}
	{\scshape\MakeLowercase{#1.}}
	[]
\titlespacing*{\section}
	{0em}
	{0em}
	{.5em}
\titleformat{\subsection}
	{\flushleft}
	{\hspace{-1.5em}\thesubsection}
	{.5em}
	{#1}
	[]
\titlespacing*{\subsection}
	{0em}
	{1\baselineskip}
	{1\baselineskip}
\titleformat{\paragraph}[runin]
	{}
	{}
	{1em}
	{#1}
	[.]
\titlespacing*{\paragraph}
	{0em}
	{0\parskip}
	{1\parskip}

\usepackage{titletoc}
% The following changes part entries
\titlecontents{part}
[1.6em]
{\vspace{.3em}}%
{\contentsmargin{0pt}}
{\contentsmargin{0pt}}
{}                 		
[\vspace{4pt}]
% The following changes chapter entries
\titlecontents{chapter}
[2.5em]
{\vspace{.3em}}%
{\contentsmargin{0pt}}
{\contentsmargin{0pt}}
{\hspace{5pt}\contentspage}                 		
[\vspace{4pt}]
% The following changes section entries
\titlecontents{section}
[5em]
{}
{\contentsmargin{0pt}}
{\contentsmargin{0pt}}
{\hspace{5pt}\contentspage}
[\vspace{5pt}]
% The following changes subsection entries
\titlecontents{subsection}
[6.5em]                              
{\vspace{-4pt}}
{\contentsmargin{0pt}\small\enspace}
{\contentsmargin{0pt}}        
{\small\contentspage}                 
[\vspace{3pt}]
% For the appendicies


\usepackage{fancyhdr}
\renewcommand{\headrulewidth}{0pt}
\setlength{\headheight}{14pt}
\pagestyle{fancy}
\renewcommand{\chaptermark}[1]{%
	\markboth{#1}{}}
\renewcommand{\sectionmark}[1]{\markright{#1}}
	\fancyhead[OL]{\sffamily\nouppercase\rightmark}
	\fancyhead[EL]{\sffamily\thepage}
	\fancyhead[ER]{\sffamily\nouppercase{\leftmark}}
	\fancyhead[OR]{\sffamily\thepage}
	\fancyfoot[L]{}
	\fancyfoot[C]{}
\fancypagestyle{plain}{%
	\fancyhead[R]{}
	\fancyhead[L]{}
	\fancyfoot[R]{}%
	\fancyfoot[L]{}
	\fancyfoot[C]{}
}

\usepackage[thmmarks]{ntheorem}
	\theoremstyle{plain}
	\theoremindent0cm
	\theorempreskip{.5\baselineskip}
	\theoremheaderfont{\upshape}
	\theorembodyfont{\normalfont}
	\theoremseparator{.}
	\newtheorem{defi}{Definición}[section]
	\newtheorem{teo}{Teorema}[section]
	\newtheorem{cor}{Corolario}[teo]
	\newtheorem{prop}{Proposición}[section]
	\newtheorem{lem}{Lema}[section]
	\theoremindent0cm
	\newtheorem{ejem}{Ejemplo}[section]
	\newtheorem{cejem}{Contraejemplo}[section]
	\newtheorem{ejer}{Ejercicio}[section]
	\theoremstyle{nonumberplain}
	\newtheorem{sol}{Solución}
	\newtheorem{obs}{Observación}
	\theoremindent0cm
	\newtheorem{proof}{Demostración}


\renewcommand*{\marginfont}{\small\sffamily}
%\renewcommand{\marginnotevadjust}{.7em}
\setlength{\marginparwidth}{2cm}

\newcounter{NumeroClase}
\newcommand{\Clase}[1]{%
	\marginnote{\refstepcounter{NumeroClase}%
		Clase Nº\,\theNumeroClase,\\ #1/2020.}%
}

\newlength{\savedparindent}
\setlength{\savedparindent}{\parindent}
\newenvironment{nota}{
\raggedright\setlength{\parindent}{\savedparindent}\commentfont
}{\par}

\makeatletter
\newcommand*{\docname}{document}
%\newcommand{\Nota}[1]{
%\begingroup
%\ifx\@currenvir\docname
%	\raggedright\setlength{\parindent}{\savedparindent}\commentfont
%    \else
%	\raggedright\sffamily
%   \fi	
%\makeatother
%#1
%\par
%\endgroup
%}
\newcommand{\Nota}[1]{\marginnote{\commentfont #1}}
\newenvironment{primerpar}{
	\fontsize{11}{13}\selectfont
}{\par\smallskip}

\newcommand{\Keyword}[1]{\textsc{#1}\index{#1}}
\newcommand{\KKeyword}[1]{\emph{#1}}

\let\mathbb=\mathbf

\renewcommand{\emptyset}{\varnothing}

%\let\oldemph\emph
%\renewcommand{\emph}{}
%\let\oldamp=\&
%\renewcommand{\&}{\textit{\oldamp}}

% LEGACY MACROS
\newcommand{\id}[1]{\mathfrak{#1}}
\newcommand{\an}[1]{\mathscr{#1}}
\newcommand{\subsq}{\subseteq}
\newcommand{\inv}{^{-1}}
\newcommand{\notall}[1]{\marginnote{\small \textcolor{darkgray}{#1}}}
\newcommand{\ho}[2]{\text{hom}(#1,#2)}
\newcommand{\hoo}[1]{\text{hom}(#1)}
\renewcommand{\t}{\times}
\newcommand{\dt}[2]{#1_1 #2 #1_2 #2 \cdots #2 #1_n}
\newcommand{\li}[2]{\lim_{#1 \to #2}}
\newcommand{\pd}[2]{\frac{\partial #1}{\partial #2}}
\newcommand{\pds}[2]{\dfrac{\partial^2 #1}{\partial #2^2}}
\newcommand{\pdx}[3]{\dfrac{\partial^2 #1}{\partial #2 \partial #3}}
\newcommand{\pol}[1]{#1_n x^n + #1_{n-1} x^{n-1} + \dots + #1_1 x + #1_0}
\newcommand{\Zn}{\mathscr{Z}/n\mathscr{Z}}
\renewcommand{\Rn}{\mathscr{R}^{\mathrm n}}
\newcommand{\R}{\mathscr{R}}
\newcommand{\Co}{\mathscr{C}}
\newcommand{\Z}{\mathscr{Z}}
\newcommand{\Q}{\mathscr{Q}}
\newcommand{\N}{\mathscr{N}}
\newcommand{\Ha}{\mathscr{H}}
\newcommand{\eb}[1]{\left\{ #1 \right\}}
