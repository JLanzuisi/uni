% This TeX file is auto generated from a Org file.
% As such, is not properly documented.
% Please refer to the Org or HTML file instead for explanations.

\hfuzz1pc
\overfullrule=2cm

\usepackage[spanish,es-noindentfirst]{babel}
\usepackage{csquotes}

\usepackage[
	includehead,
	includefoot,
	letterpaper,
	top=2cm,
	bottom=2cm,
	left=4.5cm,
	right=4.5cm,
	marginparsep=0.5cm,
	marginparwidth=4cm,
]{geometry}

\usepackage{mathtools}
\DeclareMathOperator{\Rea}{Re}
\DeclareMathOperator{\Ima}{Im}
\DeclareMathOperator{\car}{car}
\DeclareMathOperator{\traz}{tr}
\DeclareMathOperator{\gen}{gen}
\DeclareMathOperator{\mcm}{mcm}

\usepackage{unicode-math}

\setmainfont{XCharter}
\defaultfontfeatures{Scale=MatchLowercase}
\setsansfont{TeX Gyre Heros}
\setmonofont{Go Mono}

\frenchspacing
\linespread{1.04}

\setmathfont{TeX Gyre Termes Math}

\usepackage[final]{microtype}

\PassOptionsToPackage{final}{graphicx}

\usepackage{%
	xcolor,%
	graphicx,%
	cancel,%
	booktabs,
	hyphenat,
	authoraftertitle,
	pdfpages,
	metalogo
}

\usepackage[
	backend=biber,
	backref=true,
	style=trad-abbrv,
	sorting=ynt
]{biblatex}
\addbibresource{/home/jhonny/git/Misc-LaTeX-files/bib/general.bib}

\usepackage{url} 
\usepackage{hyperref} 
\hypersetup{colorlinks=true,linkcolor=black,urlcolor=black}
\usepackage[spanish,nameinlink]{cleveref}

\usepackage{enumitem} 
\setlist[enumerate]{left=-11pt,nosep}
\setlist[description]{font=\normalfont,leftmargin=\parindent}
\setlist[itemize]{label={\small\textbullet},left=-11pt}

\usepackage[final]{listings} 
\lstset{
language=R, 
numbers=left, numberstyle=\tiny\ttfamily, stepnumber=2, numbersep=5pt, 
basicstyle=\ttfamily, 
stringstyle=\ttfamily,
commentstyle=\itshape,
breaklines=true,
postbreak=\mbox{$\hookrightarrow$\enspace},
columns=flexible
}

\usepackage{caption} 
\captionsetup{
font={rm},
justification=raggedright,
singlelinecheck=false,
skip=3pt
}

\usepackage[explicit]{titlesec}

\titleformat{\part}[display]
{\flushleft\sffamily\fontsize{40}{40}\selectfont}
	{\thepart}
	{3em}
	{#1}
	[]
\titleformat{\chapter}[display]
	{\flushleft\Large\sffamily}
	{\large\thechapter}
	{1em}
	{#1}
	[]
\titlespacing*{\chapter}
	{0em}
	{0em}
	{3\baselineskip}
\titleformat{\section}[hang]
	{\flushleft\sffamily}
	{\hspace{-2.4em}\S\thesection}
	{.5em}
	{\MakeUppercase{#1}}
	[]
\titlespacing*{\section}
	{0em}
	{1.5\baselineskip}
	{0\baselineskip}
\titleformat{\subsection}
	{\flushleft\sffamily}
	{\thesubsection}
	{.5em}
	{#1}
	[]
\titlespacing*{\subsection}
	{0em}
	{1\baselineskip}
	{0\baselineskip}

\usepackage{titletoc}

\titlecontents{part}
[1em]
{\vspace{.3em}}%
{\large\contentsmargin{0pt}}
{\large\contentsmargin{0pt}}
{}                 		
[\vspace{4pt}]
\titlecontents{chapter}
[1em]
{\vspace{.3em}}%
{\contentsmargin{0pt}}
{\contentsmargin{0pt}}
{\hspace{3pt}\contentspage}                 		
[\vspace{4pt}]
\titlecontents{section}
[4em]
{}
{\contentsmargin{0pt}}
{\contentsmargin{0pt}}
{\hspace{3pt}\contentspage}
[\vspace{5pt}]
\titlecontents{subsection}
[5.5em]                              
{\vspace{-4pt}}
{\contentsmargin{0pt}\small\enspace}
{\contentsmargin{0pt}}        
{\small\contentspage}                 
[\vspace{3pt}]

\usepackage{fancyhdr}

\renewcommand{\headrulewidth}{0pt}
\setlength{\headheight}{14pt}

\pagestyle{fancy}
\renewcommand{\chaptermark}[1]{%
	\markboth{#1}{}}
\renewcommand{\sectionmark}[1]{\markright{#1}}
\fancyhead[R]{\ifodd\value{page}{\nouppercase\rightmark}\else{}\fi}
\fancyhead[L]{}
\fancyfoot[R]{\thepage}
% \fancyhead[OL]{\sffamily\nouppercase\rightmark}
% \fancyhead[EL]{\sffamily\thepage}
% \fancyhead[ER]{\sffamily\nouppercase{\leftmark}}
% \fancyhead[OR]{\sffamily\thepage}
\fancyfoot[L]{}
\fancyfoot[C]{}
\fancypagestyle{plain}{%
	\fancyhead[R]{}
	\fancyhead[L]{}
	\fancyfoot[R]{}%
	\fancyfoot[L]{}
	\fancyfoot[C]{}
}

\usepackage[thmmarks]{ntheorem}
	\theoremstyle{plain}
	\theoremindent0cm
	\theorempreskip{0cm}
	\theorempostskip{0cm}
	\theoremheaderfont{\hspace*{\parindent}\upshape}
	\theorembodyfont{\itshape}
	\theoremseparator{.}
	\newtheorem{teo}{Teorema}[section]
	\newtheorem{cor}{Corolario}[teo]
	\newtheorem{prop}{Proposición}[section]
	\newtheorem{lem}{Lema}[section]
	\theoremstyle{nonumberplain}
	\theoremheaderfont{\normalfont}
	\theorembodyfont{\upshape}
	\newtheorem{proof}{Demostración}
	\theoremstyle{plain}
	\theoremheaderfont{\upshape}
	\theorempostwork{\noindent}
	\newtheorem{definition}{Definición}[section]
