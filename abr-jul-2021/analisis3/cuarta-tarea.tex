\documentclass{scrartcl}

% So that TeX doesn't complain about small
% underfull or overfull boxes
\hfuzz1pc
% Make the overfull marker bigger
\overfullrule=2cm

% Font setup.
\usepackage{unicode-math}
\setmainfont{Reforma 1969 Blanca}
\setsansfont[Scale=MatchLowercase]{League Spartan Light}
\setmonofont[Scale=MatchLowercase]{Consolas}
\setmathfont[Scale=MatchLowercase]{KpMath-Light.otf}
% 20% bigger line height
\linespread{1.15}
% Don't put extra space after periods
\frenchspacing
\KOMAoptions{
    paper = a5,
    BCOR = 0mm,
    twoside = false,
    fontsize = {10},
    DIV = calc,
}

% Make bibliography more compact, no indents.
\KOMAoption{toc}{flat}

% Language support, usually changes between english
% and spanish.
\usepackage[spanish,es-noindentfirst]{babel}
%\usepackage[english]{babel}
\usepackage{csquotes}

% Bibliography
\usepackage[
    backend=biber,
    style=numeric-comp,
    backref=true,
    backrefstyle=two,
    abbreviate=true
]{biblatex}
\addbibresource{~/git/Misc-LaTeX-files/bib/general.bib}
\addbibresource{~/git/Misc-LaTeX-files/bib/math-books.bib}

% Graphics, mainly to insert images or
% single page PDFs.
\usepackage{graphicx}
\usepackage[dvipsnames]{xcolor}
% Handy command to typeset URLs
\usepackage{hyperref}
\hypersetup{
    colorlinks=true,
    linkcolor=Mahogany,
    filecolor=Mahogany,
    urlcolor=Black,
    citecolor=Mahogany,
}
\usepackage{url}
%\urlstyle{same}
\usepackage{metalogo}

%\usepackage{minted}

% Font style and size for title
\setkomafont{title}{\normalfont\itshape}
% Font style for the subject
\setkomafont{subject}{\normalfont\scshape}
% Font style for subtitle
\setkomafont{subtitle}{\normalfont}
\setkomafont{author}{\large}
\setkomafont{date}{\normalsize}
\setkomafont{section}{\fontseries{m}\Large}
\setkomafont{subsection}{\fontseries{m}\large}
\setkomafont{subsubsection}{\fontseries{m}\normalsize}

% Footnotes
\deffootnote{2.0em}{1.5em}{\thefootnotemark.\ }

\newcounter{exer}
\newcommand{\exercise}{%
    \stepcounter{exer}%
    \begin{center}%
        \addfontfeatures{LetterSpace=7}\large\Roman{exer}%
    \end{center}%
}
\newcommand{\solution}{
    \begin{center}
        Solución
    \end{center}
}

\newcommand{\mycopyright}{
    Copyright 2021 Jhonny Lanzuisi \url{jalb97@gmail.com}.
    This work is licensed under the Creative Commons Attribution-ShareAlike
    International (CC BY-SA 4.0)  License.
}

% CUSTOM MACROS
% math macros
\renewcommand{\Rn}{\mathbb{R}^{\mathrm{n}}}
\newcommand{\Rm}{\mathbb{R}^{\mathrm{m}}}
\newcommand{\R}{\mathbb{R}}
\newcommand{\N}{\mathbb{N}}
\newcommand{\devpart}[2]{\frac{\partial  #1}{\partial #2}}
\renewcommand{\vec}[1]{\mathbf{#1}}
\newcommand{\norm}[1]{\left\lvert #1 \right\rvert}
\newcommand{\iprod}[2]{\left\langle #1 , #2 \right\rangle}
\newcommand{\devp}[2]{\frac{\partial #1}{\partial #2}}
\DeclareMathOperator{\img}{img}
\DeclareMathOperator{\gen}{span}


\begin{document}

%
\title{Cuarta tarea}
\subtitle{Cuarta evaluación del curso}
\subject{Análisis III}
\titlehead{Universidad Simón Bolívar\hfill Caracas, Venezuela}
\author{{\normalsize hecho por} \\ Jhonny Lanzuisi\footnote{\mycopyright}}
\date{\today}
\maketitle

\exercise
Mostrar que $λ ≡ μ$ es una relación de equivalencia.

\solution
Reflexividad. Si tomamos el homeomorfimo $\varphi$ como la función identidad,
entonces claramente $\lambda = \lambda\circ\varphi$ y $\lambda\equiv\lambda$.

Simetría. Sean $\lambda,\mu$ dos caminos y supongamos que $\lambda\equiv\mu$.
Entonces existe un homeomorfismo $\varphi$ tal que $\mu = \lambda\circ\varphi$.
Como $\varphi$ es homeomorfismo tiene inversa $\varphi^{-1}$, de donde se sigue que
$\lambda = \mu\circ\varphi^{-1}$ y $\mu\equiv\lambda$.

Transitividad. Sean $\lambda,\mu,\nu$ tres caminos, y supongamos que
$\lambda\equiv\mu$ y $\mu\equiv\nu$. Entonces existen dos homeomorfismos $\varphi,\xi$
tales que $\mu = \lambda\circ\varphi$ y $\nu = \mu\circ\xi$.
Por lo tanto,
$$
\nu = \mu\circ\xi = (\lambda\circ\varphi)\circ\xi = \lambda\circ(\varphi\circ\xi),
$$
y obtenemos que $\lambda\equiv\nu$.

\exercise
Sea $λ\colon [a, b] → R^2$ el gráfico de una función $f\colon [a,b]\to\R$ de clase $C^1$ y $\omega = Adx + Bdy$
una forma diferencial continua. Muestre que
$$
\int_\lambda Ady + Bdy = \int_a^b A(t,f(t)) + B(t,f(t)) f'(t) dt.
$$

\solution
Dado que $\lambda(c) = (c,f(c))\; (c\in[a,b])$, $\lambda$ tiene que ser continua
puesto que sus funciones componentes son la identidad y $f$, ambas continuas.

La derivada $\lambda' = (1,f'(c))$ es nuevamente continua, por lo que $\lambda$
es de clase $C^1$ y por lo tanto es rectificable.

Como $\lambda$ es continua y rectificable:
\begin{align*}
\int_\lambda \omega &= \int_a^b \iprod{\omega(\lambda(c))}{\lambda'(c)} dc\\
					&= \int_a^b \iprod{\big(A(c,f(c)), B(c,f(c))\big)}{\big(1,f'(c)\big)} dc\\
					&= \int_a^b A(c,f(c)) + B(c,f(c)) f'(c)\, dc
\end{align*}


\exercise
Sean $P$ y $Q$ dos particiones de un bloque $n$−dimensional $A$ y $f\colon A → R$
una función acotada. Si $P ⊂ Q$, entonces
$$
s(f,P)\leq s(f,Q)\leq S(f,Q)\leq S(f,P).
$$

\solution
Sea $R$ un subrectángulo de $P$. Como $Q$ es un refinamiento de $P$
se tiene que existen $k$ subrectángulos de $Q$, llamémosles $U_1,\dots,U_k$, en $R$.
Es claro que $v(R) = v(U_1) + \dots + v(U_k)$.
Además, $m_R(f)\leq m_{U_i}(f)\; (0 < i \leq k)$ puesto que los valores
de $f(x)$ en $R$ incluyen todos los valores de $f(x)$ en $U_i$ y posiblemente
valores menores.
Entonces,
\begin{align*}
m_R(f)v(R) &= m_R(f)v(U_1) + \dots + m_R(f)v(U_k)\\
		   &\leq m_{U_1}(f)V(U_1) + \dots + m_{U_k}(f)v(U_k).
\end{align*}

La suma de los elementos de la derecha de la desigualdad anterior, para todos los 
subrectángulos $R$ de $P$, es $s(f,P)$ y la suma a la izquierda es $s(f,Q)$, por lo que
se obtiene el resultado buscado: $s(f,P)\leq s(f,Q)$.

Un argumento completamente análogo se puede usar cuando se consideran las sumas superiores,
para llegar a la conclusión de que $S(f,Q)\leq S(f,P)$.

\exercise
Sea $A ⊂ \Rn$ un conjunto limitado con volumen. $f\colon A → R$ es integrable
si el conjunto de discontinuidades de $f$ es finito.

\exercise
Si $f\colon A → R$ es integrable sobre $A$ y es tal que $f(x) ≥ 0$ para todo $x ∈ A$
y $ \int_A f(x) dx = 0$, entonces el conjunto
$$
\{ x\in A\colon f(x)\neq 0\}
$$
posee volumen cero.

\end{document}
