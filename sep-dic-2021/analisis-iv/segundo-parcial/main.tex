\documentclass[11pt,letter]{article}

% So that TeX doesn't complain about small
% underfull or overfull boxes
\hfuzz1pc
% Make the overfull marker bigger
\overfullrule=2cm

% Font setup.
\usepackage{unicode-math}
\setmainfont{STIX Two Text}
\setsansfont[Scale=MatchLowercase]{Source Sans 3}
\setmonofont[Scale=MatchLowercase]{Ubuntu Mono}
\setmathfont[Scale=.93]{STIX Two Math}
\linespread{1.05}
% Don't put extra space after periods
\frenchspacing
\KOMAoptions{
    paper = letter,
    BCOR = 0mm,
    twoside = false,
    fontsize = {8},
    DIV = calc,
}

% Make bibliography more compact, no indents.
\KOMAoption{toc}{flat}

% Language support, usually changes between english
% and spanish.
\usepackage[spanish,es-noindentfirst]{babel}
%\usepackage[english]{babel}
\usepackage{csquotes}

% Bibliography
\usepackage[
    backend=biber,
    style=numeric-comp,
    backref=true,
    backrefstyle=two,
    abbreviate=true
]{biblatex}
\addbibresource{~/git/Misc-LaTeX-files/bib/general.bib}
\addbibresource{~/git/Misc-LaTeX-files/bib/math-books.bib}

% Graphics, mainly to insert images or
% single page PDFs.
\usepackage{graphicx}
\usepackage[dvipsnames]{xcolor}
% Handy command to typeset URLs
\usepackage{hyperref}
\hypersetup{
    colorlinks=true,
    linkcolor=Mahogany,
    filecolor=Mahogany,
    urlcolor=Black,
    citecolor=Mahogany,
}
\usepackage{url}
%\urlstyle{same}
\usepackage{metalogo}

%\usepackage{minted}

% Font style and size for title
\setkomafont{title}{\itshape}
% Font style for the subject
\setkomafont{subject}{\normalfont}
% Font style for subtitle
\setkomafont{subtitle}{\normalfont\itshape}
\setkomafont{author}{\large}
\setkomafont{date}{\normalsize}
\setkomafont{section}{\fontseries{m}\Large}
\setkomafont{subsection}{\fontseries{m}\large}
\setkomafont{subsubsection}{\fontseries{m}\normalsize}

% Footnotes
\deffootnote{2.0em}{1.5em}{\thefootnotemark.\ }
\setkomafont{footnote}{\sffamily}

\newcounter{exer}
\newcommand{\exercise}{%
    \stepcounter{exer}%
    \begin{center}%
        \addfontfeatures{LetterSpace=7}\large\Roman{exer}%
    \end{center}%
}
\newcommand{\solution}{
    \begin{center}
        Solución
    \end{center}
}

% CUSTOM MACROS
% math macros
\renewcommand{\Rn}{\mathbb{R}^{\mathrm{n}}}
\newcommand{\Rm}{\mathbb{R}^{\mathrm{m}}}
\newcommand{\R}{\mathbb{R}}
\newcommand{\N}{\mathbb{N}}
\newcommand{\devpart}[2]{\frac{\partial  #1}{\partial #2}}
\renewcommand{\vec}[1]{\mathbf{#1}}
\newcommand{\norm}[1]{\left\lvert #1 \right\rvert}
\newcommand{\iprod}[2]{\left\langle #1 , #2 \right\rangle}
\newcommand{\devp}[2]{\frac{\partial #1}{\partial #2}}
\DeclareMathOperator{\img}{img}
\DeclareMathOperator{\gen}{span}


\errorstopmode

\begin{document}

\begin{flushleft}
	\begin{center}
		\small Análisis IV. Septiembre-Diciembre 2021. Universidad Simón Bolívar.
	\end{center}

	\vspace{2\baselineskip}
	{\Large\addfontfeatures{LetterSpace=7} SEGUNDO PARCIAL}

	\vspace{.5\baselineskip}
	Jhonny Lanzuisi.\\\today.
\end{flushleft}

\setcounter{exer}{1}

\exercise Calcule:
\[
	\int_\alpha \frac{z^7+1}{z^2(z^4+1)}.
\]
Donde $\alpha=1+9e^{it}$ con $t\in[0,2\pi]$.

\solution Dado que $\alpha(0)=\alpha(2\pi)=10$ podemos usar el teorema
del residuo para calcular la integral.

Los polos de la función se dan cuando $z^2(z^4+1)=0$, es decir,
\[
	z_1=0,\;
	z_2=\cis\frac{\pi}{4},\;
	z_3=-\cis\frac{\pi}{4},\;
	z_4=\cis\frac{-\pi}{4},\;
	z_5=-\cis\frac{-\pi}{4}.
\]

Veamos ahora los residuos de $f$ en estos puntos.

\begin{align*}
	\Res(f, z_1)&=\lim_{z\to0} \frac{d}{dz} (z^2f(z))\\
		      &=\lim_{z\to0} \frac{d}{dz} \frac{z^7+1}{z^4+1}\\
		      &=\lim_{z\to0} \frac{z^3(3z^7+7z^3-4)}{(z^4+1)^2}\\
		      &= 0.
\end{align*}

Para los demás residuos nos será útil la siguiente expresión:
\[\frac{A(z)}{B'(z)} = \frac{z^7+1}{6z^5+2z},\]
nótese que $B'$ es impar.

\[
	\Res(f, z_2) = \frac{A(z_2)}{B'(z_2)}
			   = \frac{\cis^7\frac{\pi}{4} + 1}
			   		{6\cis^5\frac{\pi}{4}+2\cis\frac{\pi}{4}}
\]

\[
	\Res(f, z_3) = \frac{A(z_3)}{B'(z_3)}
			   = \frac{-\cis^7\frac{\pi}{4} + 1}
			   		{-6\cis^5\frac{\pi}{4}-2\cis\frac{\pi}{4}}
\]

\[
	\Res(f, z_4) = \frac{A(z_4)}{B'(z_4)}
			   = \frac{\cis^7\frac{-\pi}{4} + 1}
			   		{6\cis^5\frac{-\pi}{4}+2\cis\frac{-\pi}{4}}
\]

\[
	\Res(f, z_5) = \frac{A(z_5)}{B'(z_5)}
			   = \frac{-\cis^7\frac{-\pi}{4} + 1}
			   		{-6\cis^5\frac{-\pi}{4}-2\cis\frac{-\pi}{4}}
\]

Ya tenemos todo lo que necesitamos para calcula al integral sobre $f$.

\begin{align*}
	\int_\alpha f &= 2\pi i \sum_{k=1}^5\Res(f, z_k)\\
			     &= 2\pi i \left(
			     		\frac{2\cis^7\frac{\pi}{4}}{6\cis^5\frac{\pi}{4}+2\cis\frac{\pi}{4}}
			     		+ \frac{2\cis^7\frac{-\pi}{4}}{6\cis^5\frac{-\pi}{4}+2\cis\frac{-\pi}{4}}
			     			\right)\\
			     &= 2\pi i \left(
			     		\frac{\cis^6\frac{\pi}{4}}{3\cis^4\frac{\pi}{4}+1}
			     		+ \frac{\cis^6\frac{-\pi}{4}}{3\cis^4\frac{-\pi}{4}+1}
			     			\right)\\
			     &= 2\pi i \left(
			     		\frac{\cis (2\pi)}{3\cis(\pi)+1}
			     		+ \frac{\cis(-2\pi)}{3\cis(-\pi)+1}
			     			\right)\\
			     &= 2\pi i \left(-\frac12 -\frac12\right)\\
			     &= -2\pi i.
\end{align*}

\exercise Sean $D\subset C$ un conjunto abierto y $L\subset C$ una
recta. Si $f : D\to C$ es una función contínua, la cual es analítica
en todos los puntos $z\in D, z\not\in L$ entonces f es analítica en
todo $D$.

\solution Para la demostración se hará uso del teorema de Morera.
Veremos que la integral de $f$ es cero en todos los rectángulos que
hay en $D$, con lo que tendremos que es analítica en $D$.

Podemos suponer, sin pérdida de generalidad, que $L$ es paralela al
eje real. De no ser este el caso, se puede conseguir una
transformación lineal $g$ que envíe un segmento de la recta real en
$L$. Entonces el que $f$ sea analítica implicará que $g$ es analítica
en al región correspondiente.

Queremos entonces ver que $\int_\Gamma f=0$ para todo rectángulo $D$ con
frontera $\Gamma$ cuyos lados sean paralelos al eje real o imaginario.
Para esto consideramos tres casos.

\begin{description}
	\item[$L$ no toca la región $\Gamma$.] En este caso
	$\int_\Gamma f=0$ debido a que $f$ es analítica en $D$ y
	$\Gamma$ es una curva cerrada.

	\item[Uno de los lados de $\Gamma$ coincide con $L$.]
	Llamemos $\Gamma_\epsilon$ al rectángulo que tiene los mismos
	lados que $\Gamma$ excepto por que el lado inferior (o el superior)
	esta elevado (o bajado) una cantidad $\epsilon$ en la dirección positiva del
	eje imaginario. Entonces,
	\[\int_\Gamma f = \lim_{\epsilon\to0} \int_{\Gamma_\epsilon} f,\]
	debido a la continuidad de $f$. Pero las integrales de la derecha son
	cero para cada $\epsilon$ por lo que tenemos $\int_\Gamma f=0$.

	\item[$\Gamma$ esta alrededor de $L$.] En esta caso, la recta
	$L$ divide a $\Gamma$ en dos regiones: $\Gamma_1,\Gamma_2$ cada una de
	las cuales es del caso anterior. Puesto que
	\[\int_\Gamma f=\int_{\Gamma_1} f + \int_{\Gamma_2} f,\]
	se tiene que $\int_\Gamma f=0$.
\end{description}

Entonces, por el teorema de Morera, $f$ es analítica en $D$.


\newpage
\section*{Colophon \& Copyright}
This document was typeset using \TeX%
\footnote{%
    \TeX\ is
    a typesetting software, free and open source,
    developed by Donald Knuth. \LaTeX\ is a macro
    set for \TeX\ developed by Leslie Lamport. \LuaTeX\ is
    a reworking of \TeX\ adding native support for the Lua
    programming language and unicode, among other things.
    All of them are available in all major
    operating systems.
}
and the \LaTeXe\ macros in a GNU/Linux system.
The editor used for editing the text was Visual Studio Code.%
The main typeface used is the STIX Two family%
\footnote{%
    The STIX typefaces were design for technical typesetting,
    and they include both text and math fonts.
}
for text and math,
the sans-serif typeface is Source Sans by Adobe%
\footnote{%
    This is a free and open source typeface comissioned by Adobe.
}.

\medskip
%
\begin{quote}
\sffamily\small
     E-mail: \url{jalb97@gmail.com}. \\
    Copyright (C) 2021 Jhonny Lanzuisi. \\
    This work is licensed under the Creative Commons Attri\-bu\-tion-Sha\-re\-Alike
    International (CC BY-SA 4.0)  License. To view a copy of the license,
    visit \url{https://creativecommons.org/licenses/by-sa/4.0/}.
\end{quote}

\end{document}
