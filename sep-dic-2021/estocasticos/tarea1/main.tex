\documentclass{scrartcl}

% So that TeX doesn't complain about small
% underfull or overfull boxes
\hfuzz1pc
% Make the overfull marker bigger
\overfullrule=2cm

% Font setup.
\usepackage{unicode-math}
\setmainfont{Reforma 1969 Blanca}
\setsansfont[Scale=MatchLowercase]{League Spartan Light}
\setmonofont[Scale=MatchLowercase]{Consolas}
\setmathfont[Scale=MatchLowercase]{KpMath-Light.otf}
% 20% bigger line height
\linespread{1.15}
% Don't put extra space after periods
\frenchspacing
\KOMAoptions{
    paper = a5,
    BCOR = 0mm,
    twoside = false,
    fontsize = {10},
    DIV = calc,
}

% Make bibliography more compact, no indents.
\KOMAoption{toc}{flat}

% Language support, usually changes between english
% and spanish.
\usepackage[spanish,es-noindentfirst]{babel}
%\usepackage[english]{babel}
\usepackage{csquotes}

% Bibliography
\usepackage[
    backend=biber,
    style=numeric-comp,
    backref=true,
    backrefstyle=two,
    abbreviate=true
]{biblatex}
\addbibresource{~/git/Misc-LaTeX-files/bib/general.bib}
\addbibresource{~/git/Misc-LaTeX-files/bib/math-books.bib}

% Graphics, mainly to insert images or
% single page PDFs.
\usepackage{graphicx}
\usepackage[dvipsnames]{xcolor}
% Handy command to typeset URLs
\usepackage{hyperref}
\hypersetup{
    colorlinks=true,
    linkcolor=Mahogany,
    filecolor=Mahogany,
    urlcolor=Black,
    citecolor=Mahogany,
}
\usepackage{url}
%\urlstyle{same}
\usepackage{metalogo}

%\usepackage{minted}

% Font style and size for title
\setkomafont{title}{\normalfont\itshape}
% Font style for the subject
\setkomafont{subject}{\normalfont\scshape}
% Font style for subtitle
\setkomafont{subtitle}{\normalfont}
\setkomafont{author}{\large}
\setkomafont{date}{\normalsize}
\setkomafont{section}{\fontseries{m}\Large}
\setkomafont{subsection}{\fontseries{m}\large}
\setkomafont{subsubsection}{\fontseries{m}\normalsize}

% Footnotes
\deffootnote{2.0em}{1.5em}{\thefootnotemark.\ }

\newcounter{exer}
\newcommand{\exercise}{%
    \stepcounter{exer}%
    \begin{center}%
        \addfontfeatures{LetterSpace=7}\large\Roman{exer}%
    \end{center}%
}
\newcommand{\solution}{
    \begin{center}
        Solución
    \end{center}
}

\newcommand{\mycopyright}{
    Copyright 2021 Jhonny Lanzuisi \url{jalb97@gmail.com}.
    This work is licensed under the Creative Commons Attribution-ShareAlike
    International (CC BY-SA 4.0)  License.
}

% CUSTOM MACROS
% math macros
\renewcommand{\Rn}{\mathbb{R}^{\mathrm{n}}}
\newcommand{\Rm}{\mathbb{R}^{\mathrm{m}}}
\newcommand{\R}{\mathbb{R}}
\newcommand{\N}{\mathbb{N}}
\newcommand{\devpart}[2]{\frac{\partial  #1}{\partial #2}}
\renewcommand{\vec}[1]{\mathbf{#1}}
\newcommand{\norm}[1]{\left\lvert #1 \right\rvert}
\newcommand{\iprod}[2]{\left\langle #1 , #2 \right\rangle}
\newcommand{\devp}[2]{\frac{\partial #1}{\partial #2}}
\DeclareMathOperator{\img}{img}
\DeclareMathOperator{\gen}{span}


\begin{document}
%
\title{Primera Tarea}
\subtitle{Ruina del jugador}
\subject{Aplicación a los procesos estocásticos discretos}
\titlehead{Universidad Simón Bolívar\hfill Caracas, Venezuela}
\author{Jhonny Lanzuisi}
\date{\today}
\maketitle

\exercise
Muestre que si $L\to\infty$ entonces $p_k\to1$.

\solution
Como sabemos, la probabilidad de que se vaya sin nada, $p_k$,
esta dada por:
\begin{equation}\label{eq:pk}
	p_k = \frac{
		\left(\frac{1-p}{p}\right)^L - \left(\frac{1-p}{p}\right)^k
	}{
		\left(\frac{1-p}{p}\right)^L - 1
	}
\end{equation}

Como $p < 1/2$ esta fijo, por conveniencia hagamos la siguiente sustitución.
\[
	\alpha = \frac{1-p}{p}.
\]
Nótese además que, como $p < 1/2$, se tiene $\alpha > 1$ y $\lim_{L\to\infty} \alpha^L = \infty$.

Con todo lo anterior, podemos calcular el límite buscado.
\[
	\lim_{L\to\infty} \frac{
		\alpha^L - \alpha^k
	}{
		\alpha^L - 1
	}
	=
	\lim_{L\to\infty} \frac{
		1 - \frac{\alpha^k}{\alpha^L}
	}{
		1 - \frac{1}{\alpha^L}
	}
	=
	\frac{1 - 0}{1 - 0}
	=
	1
\]

\exercise
Encuentre $p_k$ cuando $p=1/2$.

\solution
Una simple evaluación de $p=1/2$ en la ecuación~\ref{eq:pk} da una división por $0$,
por lo tanto el valor de $p_k$ puede hallarse considerando el límite cuando $p\to1/2$.

Hagamos $\alpha$ igual que en el ejercicio anterior. Entonces $p\to1/2$ implica que
$\alpha\to1$. Una evaluación del límite con la sustitución propuesta demuestra rápidamente
que se sigue teniendo una división por $0$. Hace falta, entonces, aplicar la regla de L'Hopital.
\begin{align*}
	\lim_{\alpha\to1} \frac{
		\alpha^L - \alpha^k
	}{
		\alpha^L - 1
	}
	=
	\lim_{\alpha\to1} \frac{
		L\alpha^{L-1} - k\alpha^{k-1}
	}{
		L\alpha^{L-1}
	}
	=
	\frac{L - k}{L}
	= 1 - \frac{k}{L}.
\end{align*}

\exercise
Muestre que con probabilidad uno, el jugador no se queda en
el casino por siempre.

\solution
Queremos ver que el juego siempre termina. Llamemos $q_k$ a la probabilidad
de que el jugador se queda jugando para siempre. Queremos ver que $q_k = 0$ para
todo $k$. Como aplican las mismas condiciones que para $p_k$, se tienen que la misma
recurrencia es válida:
\[
	q_k = pq_{k+1} + (1-p)q_{k-1},
\]
pero con la diferencia de que $q_0=0$, pues si tiene $0$ dinero no puede jugar para siempre,
y $q_L = 0$ pues si llega a $L$ se retira y no sigue jugando.

Con estas condiciones inciales se puede despejar $q_k$ de la misma forma que
se hizo en clases, pero tomando en cuenta las diferencias causadas por el cambio
en la condición incial. Se tiene entonces que de la recursión anterior y de la parte (i)
del teorema visto en clases se llega a la siguiente recursión.
\[
	q_{k+1}-q_k = \left(\frac{p-1}{p}\right)^k(q_1).
\]

Ahora, aplicando la parte (ii) del teorema, obtenemos:
\begin{equation}\label{eq:qk}
	q_k = (q_1) \frac{
		\left(\frac{p-1}{p}\right)^k - 1
	}{
		\frac{p-1}{p} - 1
	},
\end{equation}
que al hacer $k=L$ se tiene $q_k=0$, volviendo a la recurrencia anterior:
\[
	0 = (q_1) \frac{
		\left(\frac{p-1}{p}\right)^L - 1
	}{
		\frac{p-1}{p} - 1
	},
\]
de donde se sigue que $q_1 = 0$ o  $\frac{
		\left(\frac{p-1}{p}\right)^L - 1
	}{
		\frac{p-1}{p} - 1
}$ es cero.

La segunda posibilidad implica, despues de despejar, que $p = 1/2$ lo cual es imposible
pues estamos suponiendo que $p<1/2$. Se tiene entonces que $q_1=0$ y de~(\ref{eq:qk})
se obtiene $q_k = 0$ como se buscaba.

\exercise
Dado que se va con nada, ¿Cúal es la probabilidad de que nunca haya
tenido más de los $\$ k$ iniciales?

%\printbibliography
\pagebreak
\section*{Colophon \& Copyright}
This document was typeset using \TeX%
\footnote{%
    \TeX\ is
    a typesetting software, free and open source,
    developed by Donald Knuth. \LaTeX\ is a macro
    set for \TeX\ developed by Leslie Lamport.
},
the \LaTeXe\ macros with the pdf\TeX\ engine in a Linux system.
The editor used for editing the text was sam.
The main fonts are Helvetica and Euler math.

\begingroup\small
\medskip
%
\noindent E-mail: \url{jalb97@gmail.com}. \\
Copyright (C) 2022 Jhonny Lanzuisi. \\
This work is licensed under the Creative Commons Attri\-bu\-tion-Sha\-re\-Alike
International (CC BY-SA 4.0)  License. To view a copy of the li\-cense,
visit \url{https://creativecommons.org/licenses/by-sa/4.0/}.
\endgroup

\end{document}