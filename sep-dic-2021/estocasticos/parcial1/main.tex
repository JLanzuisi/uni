\documentclass{scrartcl}

% So that TeX doesn't complain about small
% underfull or overfull boxes
\hfuzz1pc
% Make the overfull marker bigger
\overfullrule=2cm

% Font setup.
\usepackage{unicode-math}
\setmainfont{STIX Two Text}
\setsansfont[Scale=MatchLowercase]{Source Sans 3}
\setmonofont[Scale=MatchLowercase]{Ubuntu Mono}
\setmathfont[Scale=.93]{STIX Two Math}
\linespread{1.05}
% Don't put extra space after periods
\frenchspacing
\KOMAoptions{
    paper = letter,
    BCOR = 0mm,
    twoside = false,
    fontsize = {8},
    DIV = calc,
}

% Make bibliography more compact, no indents.
\KOMAoption{toc}{flat}

% Language support, usually changes between english
% and spanish.
\usepackage[spanish,es-noindentfirst]{babel}
%\usepackage[english]{babel}
\usepackage{csquotes}

% Bibliography
\usepackage[
    backend=biber,
    style=numeric-comp,
    backref=true,
    backrefstyle=two,
    abbreviate=true
]{biblatex}
\addbibresource{~/git/Misc-LaTeX-files/bib/general.bib}
\addbibresource{~/git/Misc-LaTeX-files/bib/math-books.bib}

% Graphics, mainly to insert images or
% single page PDFs.
\usepackage{graphicx}
\usepackage[dvipsnames]{xcolor}
% Handy command to typeset URLs
\usepackage{hyperref}
\hypersetup{
    colorlinks=true,
    linkcolor=Mahogany,
    filecolor=Mahogany,
    urlcolor=Black,
    citecolor=Mahogany,
}
\usepackage{url}
%\urlstyle{same}
\usepackage{metalogo}

%\usepackage{minted}

% Font style and size for title
\setkomafont{title}{\itshape}
% Font style for the subject
\setkomafont{subject}{\normalfont}
% Font style for subtitle
\setkomafont{subtitle}{\normalfont\itshape}
\setkomafont{author}{\large}
\setkomafont{date}{\normalsize}
\setkomafont{section}{\fontseries{m}\Large}
\setkomafont{subsection}{\fontseries{m}\large}
\setkomafont{subsubsection}{\fontseries{m}\normalsize}

% Footnotes
\deffootnote{2.0em}{1.5em}{\thefootnotemark.\ }
\setkomafont{footnote}{\sffamily}

\newcounter{exer}
\newcommand{\exercise}{%
    \stepcounter{exer}%
    \begin{center}%
        \addfontfeatures{LetterSpace=7}\large\Roman{exer}%
    \end{center}%
}
\newcommand{\solution}{
    \begin{center}
        Solución
    \end{center}
}

% CUSTOM MACROS
% math macros
\renewcommand{\Rn}{\mathbb{R}^{\mathrm{n}}}
\newcommand{\Rm}{\mathbb{R}^{\mathrm{m}}}
\newcommand{\R}{\mathbb{R}}
\newcommand{\N}{\mathbb{N}}
\newcommand{\devpart}[2]{\frac{\partial  #1}{\partial #2}}
\renewcommand{\vec}[1]{\mathbf{#1}}
\newcommand{\norm}[1]{\left\lvert #1 \right\rvert}
\newcommand{\iprod}[2]{\left\langle #1 , #2 \right\rangle}
\newcommand{\devp}[2]{\frac{\partial #1}{\partial #2}}
\DeclareMathOperator{\img}{img}
\DeclareMathOperator{\gen}{span}


\begin{document}
%
\title{Primer Parcial}
\subtitle{Paseos aleatorios \& cadenas de Markov}
\subject{Aplicación a los procesos estocásticos discretos}
\titlehead{Universidad Simón Bolívar\hfill Caracas, Venezuela}
\author{Jhonny Lanzuisi}
\date{\today}
\maketitle

\exercise
Sea $S_n$ con $n ∈ \N$ un paseo al azar simétrico. Dado $n ∈ \N$, ¿cuál es la probabilidad
de que $S$ no visite el cero durante el intervalo $(0, n]$?.

\solution
Si $n$ es par, entonces el lema principal asegura que:
\[
	\Prob(S_1\neq0, \dots, S_n\neq0) = \Prob(S_n=0).
\]
Además, por lo visto en clases,
\[
	\Prob(S_n=0) = \binom{2n}{n}\frac{1}{2^{2n}}.
\]

Si $n$ es impar, podemos tomar el primer par $m$ tal que $m<n$.
Como el paseo no puede hacerse cero en $n$ impar, obtendremos la probabilidad
buscada usando las mismas igualdades de antes pero con $m$ en lugar de $n$.

\exercise
Sea $\mathcal{X}$ una cadena de Markov de tres estados, $E = \{1, 2, 3\}$, dada por la matriz
de transición
\[
	P =
	\begin{pmatrix}
		0.3 & 0.2 & 0.5 \\
		0.5 & 0.1 & 0.4 \\
		0.5 & 0.2 & 0.3
	\end{pmatrix}	
\]

\begin{enumerate}[label=\alph*]
\item Halle $\Prob(X_2 = 3 \mid X_0 = 1)$.
\item Halle la distribución de $π_1$ , es decir, la probabilidad de que para el paso uno la cadena esté en cada estado.
\item Halle la distribución estacionaria de la cadena, es decir, $π$ tal que $πP = π$.
\item Encuentre $\Prob(X_0 = 3, X_1 = 2, X_2 = 1, X_3 = 1, X_4 = 3)$.
\end{enumerate}

\solution[a]
Por un teorema visto en clases,
\[
	\Prob(X_{0+2} = 3\mid X_0 = 1) = P^2(1,3).
\]
Calculando el cuadrado de la matriz $P$, obtenemos,
\[
	P^2 =
	\begin{pmatrix}	
		0.44 & 0.18 & 0.38 \\
		0.4 & 0.19 & 0.41 \\
		0.4 & 0.18 & 0.42
	\end{pmatrix},
\]
de donde se sigue que $\Prob(X_{0+2} = 3\mid X_0 = 1) = 0.38$.

\clearpage
\solution[d]
Por un teorema visto en clases, se tiene que:
\begin{align*}
	\Prob(X_0 = 3, X_1 = 2, X_2 = 1, X_3 = 1, X_4 = 3) &= \pi_3 p_{32} p_{21} p_{11} p_{13}\\
	&= (\frac13) (0.2) (0.5) (0.3) (0.5)\\
	&= 0.005.
\end{align*}

\solution[c]
Queremos ver las soluciones al sistema de ecuaciones $\pi P = \pi$, dado por:
\[
	\begin{cases}
		x_1(0.3) + x_2(0.5) + x_3(0.5) = x_1\\
		x_1(0.2) + x_2(0.1) + x_3(0.2) = x_2\\
		x_1(0.5) + x_2(0.4) + x_3(0.3) = x_3,
	\end{cases}
\]
bajo la condición adicional $x_1 + x_2 + x_3 = 1$ que debe cumplir
la distribución de probabilidad.

El sistema anterior tiene como solución:
\[
	x_1 = \frac{55}{53}x_3, \quad x_2 = \frac{24}{53}x_3, \quad x_3=x_3.	
\]

Ahora podemos despejar $x_3$ de la ecuación:
\[
	\frac{55}{53}x_3 + \frac{24}{53}x_3	+ x_3 = 1
\]
y obtenemos $x_3 = 0.40152$.

Se sigue que los valores buscados de $\pi$ son $(0.41667, 0.18182, 0.40152)$.

\clearpage
\solution[b]
La distribución $\pi_1$ viene dada por:
\[
	(1/3, 1/3, 1/3)
	\begin{pmatrix}
		0.3 & 0.2 & 0.5 \\
		0.5 & 0.1 & 0.4 \\
		0.5 & 0.2 & 0.3
	\end{pmatrix}
	= (0.4333, 0.1667, 0.4).
\]

\exercise
Considere una cadena de Markov con espacio de estados $E = \{0, 1, 2, 3\}$ y matriz
de transición
\[
	P = \begin{pmatrix}
		1 & 0 & 0 & 0 \\
		0.5 & 0.2 & 0.1 & 0.2\\
		0.2  & 0.1 & 0.6 & 0.1\\
		0 & 0 & 0 & 1 \\
		\end{pmatrix}.
\]

\begin{enumerate}[label=\alph*]
	\item Comenzando en el estado 1, determine la probabilidad de que la cadena de Markov termine en el estado 0.
	\item Determine el tiempo medio de absorción.
\end{enumerate}

\solution[a]
Sea $h_i = \Prob(\text{llegar al 0, partiendo del estado $i$})$.
Las probabilidades de llegar al estado cero partiendo de los otros estados
vienen dadas por el siguiente sistema linear.
\[
	\begin{cases}
		h_0 = 1\\
		h_3 = 0\\
		h_1 = (0.5)h_0 + (0.1)h_2 + (0.2)h_1 + (0.2)h_3\\
		h_2 = (0.1)h_1 + (0.1)h_3 + (0.2)h_0 + (0.6)h_2 \\
	\end{cases}	
\]

Las soluciones del sistema anterior son $h_2=0.677$ y $h_1 = 0.709$.
De estas, $h_1$ es la probabilidad buscada.

\solution[b]
Sea $k_i = E(\text{tiempo en llegar al $\{0,3\}$ desde el estado $i$})$.
Los tiempos estimados para llegar al estado cero partiendo de los otros estados
vienen dadas por el siguiente sistema linear.
\[
	\begin{cases}
		k_0 = 0\\
		k_3 = 0\\
		k_1 = (0.5)k_0 + (0.1)k_2 + (0.2)k_1 + (0.2)k_3 + 1\\
		k_2 = (0.1)k_1 + (0.1)k_3 + (0.2)k_0 + (0.6)k_2 + 1\\
	\end{cases}	
\]

Las soluciones del sistema anterior son $k_2=2.9$ y $k_1 = 1.61$.

\newpage
\section*{Colophon \& Copyright}
This document was typeset using \TeX%
\footnote{%
    \TeX\ is
    a typesetting software, free and open source,
    developed by Donald Knuth. \LaTeX\ is a macro
    set for \TeX\ developed by Leslie Lamport. \LuaTeX\ is
    a reworking of \TeX\ adding native support for the Lua
    programming language and unicode, among other things.
    All of them are available in all major
    operating systems.
}
and the \LaTeXe\ macros in a GNU/Linux system.
The editor used for editing the text was Visual Studio Code.%
The main typeface used is the STIX Two family%
\footnote{%
    The STIX typefaces were design for technical typesetting,
    and they include both text and math fonts.
}
for text and math,
the sans-serif typeface is Source Sans by Adobe%
\footnote{%
    This is a free and open source typeface comissioned by Adobe.
}.

\medskip
%
\begin{quote}
\sffamily\small
     E-mail: \url{jalb97@gmail.com}. \\
    Copyright (C) 2021 Jhonny Lanzuisi. \\
    This work is licensed under the Creative Commons Attri\-bu\-tion-Sha\-re\-Alike
    International (CC BY-SA 4.0)  License. To view a copy of the license,
    visit \url{https://creativecommons.org/licenses/by-sa/4.0/}.
\end{quote}

\end{document}