\documentclass{scrartcl}

% So that TeX doesn't complain about small
% underfull or overfull boxes
\hfuzz1pc
% Make the overfull marker bigger
\overfullrule=2cm

% Font setup.
\usepackage{unicode-math}
\setmainfont{STIX Two Text}
\setsansfont[Scale=MatchLowercase]{Source Sans 3}
\setmonofont[Scale=MatchLowercase]{Ubuntu Mono}
\setmathfont[Scale=.93]{STIX Two Math}
\linespread{1.05}
% Don't put extra space after periods
\frenchspacing
\KOMAoptions{
    paper = letter,
    BCOR = 0mm,
    twoside = false,
    fontsize = {8},
    DIV = calc,
}

% Make bibliography more compact, no indents.
\KOMAoption{toc}{flat}

% Language support, usually changes between english
% and spanish.
\usepackage[spanish,es-noindentfirst]{babel}
%\usepackage[english]{babel}
\usepackage{csquotes}

% Bibliography
\usepackage[
    backend=biber,
    style=numeric-comp,
    backref=true,
    backrefstyle=two,
    abbreviate=true
]{biblatex}
\addbibresource{~/git/Misc-LaTeX-files/bib/general.bib}
\addbibresource{~/git/Misc-LaTeX-files/bib/math-books.bib}

% Graphics, mainly to insert images or
% single page PDFs.
\usepackage{graphicx}
\usepackage[dvipsnames]{xcolor}
% Handy command to typeset URLs
\usepackage{hyperref}
\hypersetup{
    colorlinks=true,
    linkcolor=Mahogany,
    filecolor=Mahogany,
    urlcolor=Black,
    citecolor=Mahogany,
}
\usepackage{url}
%\urlstyle{same}
\usepackage{metalogo}

%\usepackage{minted}

% Font style and size for title
\setkomafont{title}{\itshape}
% Font style for the subject
\setkomafont{subject}{\normalfont}
% Font style for subtitle
\setkomafont{subtitle}{\normalfont\itshape}
\setkomafont{author}{\large}
\setkomafont{date}{\normalsize}
\setkomafont{section}{\fontseries{m}\Large}
\setkomafont{subsection}{\fontseries{m}\large}
\setkomafont{subsubsection}{\fontseries{m}\normalsize}

% Footnotes
\deffootnote{2.0em}{1.5em}{\thefootnotemark.\ }
\setkomafont{footnote}{\sffamily}

\newcounter{exer}
\newcommand{\exercise}{%
    \stepcounter{exer}%
    \begin{center}%
        \addfontfeatures{LetterSpace=7}\large\Roman{exer}%
    \end{center}%
}
\newcommand{\solution}{
    \begin{center}
        Solución
    \end{center}
}

% CUSTOM MACROS
% math macros
\renewcommand{\Rn}{\mathbb{R}^{\mathrm{n}}}
\newcommand{\Rm}{\mathbb{R}^{\mathrm{m}}}
\newcommand{\R}{\mathbb{R}}
\newcommand{\N}{\mathbb{N}}
\newcommand{\devpart}[2]{\frac{\partial  #1}{\partial #2}}
\renewcommand{\vec}[1]{\mathbf{#1}}
\newcommand{\norm}[1]{\left\lvert #1 \right\rvert}
\newcommand{\iprod}[2]{\left\langle #1 , #2 \right\rangle}
\newcommand{\devp}[2]{\frac{\partial #1}{\partial #2}}
\DeclareMathOperator{\img}{img}
\DeclareMathOperator{\gen}{span}


\begin{document}
%
\title{Cuarta Tarea}
\subtitle{Procesos de Poisson y nacimiento}
\subject{Aplicación a los procesos estocásticos discretos}
\titlehead{Universidad Simón Bolívar\hfill Caracas, Venezuela}
\author{Jhonny Lanzuisi}
\date{\today}
\maketitle

\section*{Proceso de Poisson}
\exercise
Sean $X$ y $Y$ variables aleatorias independientes distribuidas Poisson con parámetros $α$ y $β$, 
respectivamente. Determine la distribución condicional de $X$, dado que $N = X + Y = n$.

\solution
Sabemos que: 
\begin{gather*}
    p_X(j) = P\{X=j\} = \frac{\alpha^j e^{-\alpha}}{j!},\\
    p_Y(j) = P\{Y=j\} = \frac{\beta^j e^{-\beta}}{j!},\\
    p_{X+Y}(j) = P\{{X+Y}=j\} = \frac{(\alpha+\beta)^j e^{-(\alpha+\beta)}}{j!}.
\end{gather*}

Buscamos la función $p_{X\mid N}$, dada por:
\begin{align*}
    p_{X\mid N}(k\mid n) &= P\{X = k\mid N = n\}\\
                         &= \frac{P\{X = k\mid Y = n-k\}}{P\{N = n\}}\\
                         &= \frac{p_X(k) p_Y(n-k)}{p_{X+Y}(n)}\\
                         &= \frac{\alpha^k e^{-\alpha}}{k!}\frac{\beta^{n-k} e^{-\beta}}{(n-k)!}
                                \bigg/ \frac{(\alpha+\beta)^n e^{-(\alpha+\beta)}}{n!}.
\end{align*}

\section*{Proceso de nacimiento y muerte}
\setcounter{exer}{0}
\exercise
Un proceso de nacimiento puro comienza desde $X(0) = 0$ y tiene parámetros de nacimiento 
$λ_0 = 1, λ_1 = 3, λ_2 = 2, λ_3 = 5$. Determine $P_n (t)$ para $n = 0, 1, 2, 3$.

\solution
Como se tiene un proceso de nacimiento puro $\mu_i = 0 \,\forall i$.
\[
    P'_n(t) = -\lambda_n P_n(t) + \lambda_{n-1} P_{n-1}(t).
\]

Hagamos $n=1$. 
\[
  P'_1(t) = -\lambda_1 P_1(t) + \lambda_0 P_{0}(t) = 3P_1(t) + 1.
\]
La solución a la anterior ecuación diferencial es $P_1 = e^{3t}/3 - 1/3$.

Hagamos $n=2$. 
\[
  P'_2(t) = -2 P_2(t) + 3 e^{3t}/3 - 1/3
\]
La solución a la anterior ecuación diferencial es $P_1 = e^{3t}/3 - 1/3$.
\newpage
\section*{Colophon \& Copyright}
This document was typeset using \TeX%
\footnote{%
    \TeX\ is
    a typesetting software, free and open source,
    developed by Donald Knuth. \LaTeX\ is a macro
    set for \TeX\ developed by Leslie Lamport. \LuaTeX\ is
    a reworking of \TeX\ adding native support for the Lua
    programming language and unicode, among other things.
    All of them are available in all major
    operating systems.
}
and the \LaTeXe\ macros in a GNU/Linux system.
The editor used for editing the text was Visual Studio Code.%
The main typeface used is the STIX Two family%
\footnote{%
    The STIX typefaces were design for technical typesetting,
    and they include both text and math fonts.
}
for text and math,
the sans-serif typeface is Source Sans by Adobe%
\footnote{%
    This is a free and open source typeface comissioned by Adobe.
}.

\medskip
%
\begin{quote}
\sffamily\small
     E-mail: \url{jalb97@gmail.com}. \\
    Copyright (C) 2021 Jhonny Lanzuisi. \\
    This work is licensed under the Creative Commons Attri\-bu\-tion-Sha\-re\-Alike
    International (CC BY-SA 4.0)  License. To view a copy of the license,
    visit \url{https://creativecommons.org/licenses/by-sa/4.0/}.
\end{quote}

\end{document}
