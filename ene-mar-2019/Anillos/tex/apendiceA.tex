%\input{../../Plantillas-Fomato/Libros/Libro-Anillos}
\chapter{Sobre \LaTeX{} y esta guía}
%\cita{Is typography an art? That’s like asking if photography is an art (\ldots) at their core, both photography and typography perform a utilitarian function. The aesthetic component is separate.}{Matthew Butterick}
{\noindent Este apéndice pretende lograr dos cosas: explicar, mas o menos, como fue hecha esta guía ---para aquellos interesados en estas cosas--- y decir una o dos cosas sobre tipografía.}

Primero, ¿cómo se hizo esta guía? La respuesta a la pregunta es fácil: con \LaTeX. Mas específicamente, con \TeX Live y \TeX Studio. Para el que no tenga idea de que son estas palabras mágicas, la explicación es sencilla, \LaTeX{} es un sistema de tipografía para computadoras y es además software \textit{libre}. 

\Nota{Para los que no sepan sobre el software libre, revisen \url{https://www.fsf.org/about/what-is-free-software}.}
%
Como es software libre, existen varias \textit{distribuciones} de ---for\-mas de empaquetar y distribuir--- \LaTeX , \TeX Live es una de ellas. \TeX Studio, por otra parte, es un editor de texto. Vale la pena decir que, teniendo en cuenta lo variado que es el mundo de los editores de texto, cualquiera sirve para crear documentos en \LaTeX : desde uno simple como \texttt{GNU nano} hasta algunos mas grandes y complejos. 

En lo que respecta a la tipografía de esta guía, hay algunas cosas que me gustaría decir: cuales son las fuentes y una que otra cosa acerca del formato.

Para obtener el código fuente de este documento, y tener así todos los demás detalles, véase \url{ALGOAKI}. De no funcionar el link, pueden enviarme un correo a la dirección que esta en el prefacio. 

Por último, y en lo que respecta a la tipografía en general, daré dos recomendaciones rápidas y unas cuantas referencias bibliográficas. La primera recomendación es usar márgenes grandes, pues ayudan a leer mejor, y la segunda es ser \textit{minimalista}: usar la menor cantidad de herramientas para obtener el efecto deseado en el texto. Para los interesados en el tema, que tiene mucho de interesante, recomiendo el libro-web de Butterick~\cite{typobutterick} y el libro ---físico, pero se puede encontrar en \url{libgen.io}--- de Bringhurst~\cite{typobringhurst}.
