\documentclass{scrartcl}

% So that TeX doesn't complain about small
% underfull or overfull boxes
\hfuzz1pc
% Make the overfull marker bigger
\overfullrule=2cm

% Font setup.
\usepackage{unicode-math}
\setmainfont{Reforma 1969 Blanca}
\setsansfont[Scale=MatchLowercase]{League Spartan Light}
\setmonofont[Scale=MatchLowercase]{Consolas}
\setmathfont[Scale=MatchLowercase]{KpMath-Light.otf}
% 20% bigger line height
\linespread{1.15}
% Don't put extra space after periods
\frenchspacing
\KOMAoptions{
    paper = a5,
    BCOR = 0mm,
    twoside = false,
    fontsize = {10},
    DIV = calc,
}

% Make bibliography more compact, no indents.
\KOMAoption{toc}{flat}

% Language support, usually changes between english
% and spanish.
\usepackage[spanish,es-noindentfirst]{babel}
%\usepackage[english]{babel}
\usepackage{csquotes}

% Bibliography
\usepackage[
    backend=biber,
    style=numeric-comp,
    backref=true,
    backrefstyle=two,
    abbreviate=true
]{biblatex}
\addbibresource{~/git/Misc-LaTeX-files/bib/general.bib}
\addbibresource{~/git/Misc-LaTeX-files/bib/math-books.bib}

% Graphics, mainly to insert images or
% single page PDFs.
\usepackage{graphicx}
\usepackage[dvipsnames]{xcolor}
% Handy command to typeset URLs
\usepackage{hyperref}
\hypersetup{
    colorlinks=true,
    linkcolor=Mahogany,
    filecolor=Mahogany,
    urlcolor=Black,
    citecolor=Mahogany,
}
\usepackage{url}
%\urlstyle{same}
\usepackage{metalogo}

%\usepackage{minted}

% Font style and size for title
\setkomafont{title}{\normalfont\itshape}
% Font style for the subject
\setkomafont{subject}{\normalfont\scshape}
% Font style for subtitle
\setkomafont{subtitle}{\normalfont}
\setkomafont{author}{\large}
\setkomafont{date}{\normalsize}
\setkomafont{section}{\fontseries{m}\Large}
\setkomafont{subsection}{\fontseries{m}\large}
\setkomafont{subsubsection}{\fontseries{m}\normalsize}

% Footnotes
\deffootnote{2.0em}{1.5em}{\thefootnotemark.\ }

\newcounter{exer}
\newcommand{\exercise}{%
    \stepcounter{exer}%
    \begin{center}%
        \addfontfeatures{LetterSpace=7}\large\Roman{exer}%
    \end{center}%
}
\newcommand{\solution}{
    \begin{center}
        Solución
    \end{center}
}

\newcommand{\mycopyright}{
    Copyright 2021 Jhonny Lanzuisi \url{jalb97@gmail.com}.
    This work is licensed under the Creative Commons Attribution-ShareAlike
    International (CC BY-SA 4.0)  License.
}

% CUSTOM MACROS
% math macros
\renewcommand{\Rn}{\mathbb{R}^{\mathrm{n}}}
\newcommand{\Rm}{\mathbb{R}^{\mathrm{m}}}
\newcommand{\R}{\mathbb{R}}
\newcommand{\N}{\mathbb{N}}
\newcommand{\devpart}[2]{\frac{\partial  #1}{\partial #2}}
\renewcommand{\vec}[1]{\mathbf{#1}}
\newcommand{\norm}[1]{\left\lvert #1 \right\rvert}
\newcommand{\iprod}[2]{\left\langle #1 , #2 \right\rangle}
\newcommand{\devp}[2]{\frac{\partial #1}{\partial #2}}
\DeclareMathOperator{\img}{img}
\DeclareMathOperator{\gen}{span}


\begin{document}
%
\newcommand{\tpak}{`Train to pakistan'}
\newcommand{\ks}{Kushwant Singh}
\title{Essay on \tpak}
\subtitle{Answer to question No. 2}
\subject{IDE 143---El mundo de la literatura en inglés}
\titlehead{Universidad Simón Bolívar\hfill Caracas, Venezuela}
\author{by \\ Jhonny Lanzuisi}
\date{\today}
\maketitle

\section*{After reading \tpak\ what effects of Partition are evident in the story. How has Partition effected the lives of the people in the village?}
\label{sec:question}

In his 1956 novel \tpak \cite{khushwant_singh_train_1990},
author \ks\ portraits the period of Indian history know
as ``The partition of 1947'' trough the lens of a small village
with mixed religious population.

The \( i = 3yz^2\) author uses this setting to show and criticize
the partition period, because of it's inhumanity and
atrocities.

\[ \sum x^2 p = \frac12 qr^r\]

Racism, religious and national fanaticism,
public slaughter and generalized violence against ethnic groups;
are all portrayed by \ks, sometimes in a subtle way,
other times abrupt and disheartening.

The overall atmosphere of the history
helps emphasize this elements,
it is mostly a gloomy and sad atmosphere
with clouded skies.

One of the most important effects of the Partition
that also plays a major roll in \tpak is
the hatred and distrust that can quickly arise
between peoples that, not so long ago,
were living peacefully.

In the novel we see the history of a village
being taken apart.
First the villagers seem to have fallen victims
to distrust and internal conflict:
\begin{quote}
  Our problem is: \emph{what are we to do with all
    these pigs we have with us?} (...) We have treated them
  like our own brothers. \emph{They have behaved like snakes}.
  \cite[p. 171, para. 6, emphasis added by me]{khushwant_singh_train_1990}.
\end{quote}
The Sikh population of the village,
having heard the tales of horror coming from Pakistan,
begin fearing and being angry at the Muslim part of the village.

This, in a heartwarming scene, is resolved
by the Muslims interrupting the reunion
thus making the Sikhs realize their errors
and they decide to help each other instead.

In this part of the story a new effect of the
partition comes into play.
The Muslims must go from the village by decree,
that's what the partition is all about,
to Pakistan.
This means that the gained friendship is destined
to be doomed from the beginning
and that, ultimately,
the will and decisions of the villagers are meaningless.
At the end the Muslims will be forced to leave or
face dangerous consequences.

This means that the partition established a system
of almost forced removals, given that the alternative
to going away from one's home would likely be death.

Two more elements are left to be portrayed
before the end of the story:
the struggle of women and the great levels of poverty.
At the end we'll be treated to a final element of the partition
period, the last one to be shown in \tpak.

Trough the character of a young Muslim women,
who's in love with a Sikh men and carries their child,
we see the hard times women went trough and
the destruction of many human relationships.

It is not only the fact that women were
likely to be raped or violently treated in various ways,
but also the fact that being forced to move would
make it so that many would lose their loved ones.
In the case of our Muslim women, also a father for her child.

The situation of women is then a dire one,
whether they stay or they leave they're destined
to great suffering.

To all this we have to add the fact that the villagers
are poor farmers and country men and women.
The little they have they've keep for generations,
and they'll likely inherit it to their children.
This means they don't keep their saving in a bank,
or even in money.
All they have is material in it's most raw form.

This of course contradicts with their impeding
transport to Pakistan.
They cannot take they belongings with them,
so they must loose the little they had.

With this, author \ks, introduces another aspect of
the partition: the dishonest and wrongful distribution
of property in the forcefully leaved areas.
In the story, a group of newcomers (Sikh refugees)
get to effectively steal all Muslim property.

At the end of the story, as the Muslims get prepared
to board the jeeps that'll take them away,
we see our last image from the partition, and
a very sorrowful one:
\begin{quote}
  In the confusion of rain, mud, and soldiers
  herding the peasants with the muzzles of their sten guns
  sticking in their backs, the villagers saw little of each other.
  (...)
  The skis watched them until they were out of sight,
  they watched their tears of their faces and turned back to
  their homes with heavy hearts.
  \cite[p. 185, para. 1]{khushwant_singh_train_1990}
\end{quote}
This last element of the partition is the treatment
of human being like live stock, to be moved and displaced
at the will of some foreign peoples,
living far away.

As we've seen so far,
\ks\ trough his work gives us a window
to look at the partition of India
by the imperial British,
to witness it's horrors and,
hopefully,
also to learn from them.
%
\printbibliography
\pagebreak
\section*{Colophon \& Copyright}
This document was typeset using \TeX%
\footnote{%
    \TeX\ is
    a typesetting software, free and open source,
    developed by Donald Knuth. \LaTeX\ is a macro
    set for \TeX\ developed by Leslie Lamport.
},
the \LaTeXe\ macros with the pdf\TeX\ engine in a Linux system.
The editor used for editing the text was sam.
The main fonts are Helvetica and Euler math.

\begingroup\small
\medskip
%
\noindent E-mail: \url{jalb97@gmail.com}. \\
Copyright (C) 2022 Jhonny Lanzuisi. \\
This work is licensed under the Creative Commons Attri\-bu\-tion-Sha\-re\-Alike
International (CC BY-SA 4.0)  License. To view a copy of the li\-cense,
visit \url{https://creativecommons.org/licenses/by-sa/4.0/}.
\endgroup

\end{document}
