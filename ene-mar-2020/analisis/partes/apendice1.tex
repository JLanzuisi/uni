\chapter{Colofón}%
\label{cha:Colofón}

%\begin{primerpar}
	\textsc{La fuente príncipal del texto} es \textquote{Látín M\" odern Roman} a 11pt, junto con Red Hat Sans como acompañante para los comentarios,
	Time New Roman para los títulos y Latin Modern Math para las matemáticas. Una explicación mas a fondo de cada una de estas fuentes
	junto con la implementación en \LuaLaTeX{} puede encontrarse más abajo.
\Nota{¡Todas las fuentes utilizadas se pueden obtener de forma gratuita!
	Esto muestra que se puede hacer buena tipografía sin gastar dinero
	(aquellos que sepan lo que puede llegar a costar una familia completa
	de tipos gráficos entenderán lo que quiero decir).
}

\section*{Tipos de letra principales}%
\label{sec:Tipos de letra principales}

\begin{description}
\item[Latin Modern Roman]  es un tipo gráfico de estilo \emph{Racional} diseñado por Donald Knuth y adaptado al formato \texttt{otf}
por Bogusław Jackowski \& Janusz M. Nowacki de \textsc{gust}. 

\Nota{Donald Knuth es también el creador de \TeX{} y un computista excepcional. El lector puede ver~\cite{graham_concrete_1994}
 para convercerse de que Donald es también un gran matemático.
 `\textsc{gust}' es un grupo de usuarios polacos de \TeX{}. Para más información véase su página web \autocite{latinModernFonts_website}.}

Para hacer estas notas de usó el programa \TeX{} con el formato \LaTeXe{} y el motor \LuaTeX{} que permite usar fuentes en formato
\texttt{otf}. En este contexto, el código utilizado es:

\begin{small}
\begin{verbatim}
\usepackage{unicode-math} % Interfaz para cargar fuentes
\setmainfont[ % Fuente principal
	SmallCapsFont={Latin Modern Roman Caps},
	SmallCapsFeatures={LetterSpace=5},
	Numbers={OldStyle},
]{Latin Modern Roman}
\end{verbatim}

\end{small}

\item[Red Hat Sans] es una sans-serif de estilo \emph{geométrico} usada por la compañía Red Hat:
	\begin{quote}\small
	Red Hat Display is made for headlines and big statements. It’s an embodiment of our brand’s personality — open, straightforward, rational, but friendly and approachable, too. The Display styles have even strokes and tight spacing, with tall, open letterforms.
	\footnote{\url{https://www.redhat.com/en/about/brand/standards/typography}}
	\end{quote}

El código usado es simplemente:

\begin{small}
\begin{verbatim}
\setsansfont[Scale=MatchLowercase]{Inria Sans Light}
\end{verbatim}
\end{small}

\item[Times New Roman] no necesita presentación. Diseñada por Stanley Morrison para el periodico inglés The Times (Ver \cref{fig:timesNewRoman}) es quizás
	el tipo gráfico mas usado en el mundo.



	\begin{verbatim}
\newfontfamily{\dispfont}[
Scale=MatchLowercase,
SmallCapsFeatures={LetterSpace=7},
Numbers={OldStyle},
]{STEP}		
	\end{verbatim}
\end{description}
\begin{figure}
\includegraphics[width=.75\linewidth]{pictures/times.png}
\end{figure}

\section*{Matemáticas}%
\label{sec:Matemáticas}
Las matemáticas usan \emph{Latin Modern Math} fuente derivada de Latin Modern y diseñada originalmente
por Donald Knuth para \TeX{}.

\section*{Tipos Secundarios}%
\label{sec:Tipos Secundarios}

Este documento, afortunadamente, no requiere de casi ningún tipo secundario. El único, que
se usa esporádicamente en el documento, es \textsc{ibm} Plex Mono una fuente monoespaciada
diseñada para la compañia \textsc{ibm} por Mike Ab­bink \& Paul van der Laan
que se puede ver en los bloques de código.
\begin{small}
\begin{verbatim}
\setmonofont[Scale=MatchLowercase]{IBM Plex Mono}
\end{verbatim}	
\end{small}



%\begin{small}
%\begin{verbatim}
%% Paquete para usar fuentes en formato otf
%\usepackage{unicode-math}
%% Cargar eb garamond
%\setmainfont[
	%SmallCapsFeatures={LetterSpace=5}, % Espaciado de las versalitas
	%Numbers={OldStyle}, % Figuras de texto por defecto
%]{EB Garamond}
%% Cargar yasebau
%\setsansfont[
	%Scale=MatchLowercase
%]{Ysabeau}
%% Fuente monoespaciada
%\setmonofont[
	%Scale=MatchLowercase
%]{Latin Modern Mono}
%% Fuente de mate
%\setmathfont[
	%Scale=MatchLowercase,
%]{Latin Modern Math}
%% Tomar 'prestado' el alfabeto bb de stix
%\setmathfont[
	%range=bb,
	%Scale=1.05
%]{XITS Math}
%\end{verbatim}
%\end{small}

