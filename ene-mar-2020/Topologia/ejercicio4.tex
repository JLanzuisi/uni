\documentclass[fleqn,leqno,letterpaper,10pt,final]{article}
% Misc TeX Settings
\hfuzz1pc
\overfullrule=2cm

% Language Setup
\usepackage{polyglossia}
\setmainlanguage[spanishoperators=all,]{spanish}
\PolyglossiaSetup{spanish}{,indentfirst=false}
\usepackage{csquotes} 

% Page dimensions
\usepackage[
	includehead,
	includefoot,
	top=1.5cm,
	bottom=1.5cm,
	left=2.5cm,
	right=8.5cm,
	marginparsep=0.5cm,
	marginparwidth=7cm,
]{geometry}

% Math
\usepackage{mathtools} 
\DeclareMathOperator{\Rea}{Re}
\DeclareMathOperator{\Ima}{Im}
\DeclareMathOperator{\car}{car}
\DeclareMathOperator{\traz}{tr}
\DeclareMathOperator{\gen}{gen}
\DeclareMathOperator{\mcm}{mcm}
\DeclareMathOperator{\id}{id}

% Text Fonts
\usepackage{unicode-math} 
\defaultfontfeatures{SmallCapsFeatures={LetterSpace=10},Scale=MatchLowercase,Numbers={OldStyle,Proportional}}
\setmainfont[
SmallCapsFont=* Caps,
]{Latin Modern Roman}
\setsansfont{Libertinus Sans}
\setmonofont[BoldFont=* Semibold]{Source Code Pro}
\newcommand{\dispfont}{} 
\frenchspacing
\setlength{\parindent}{1em}
\linespread{1.02}

% Microtype
\usepackage[
final,
% tracking=smallcaps,
% expansion=alltext,
% protrusion=true
]{microtype}
% \SetTracking{encoding=*,shape=sc}{50}

% Misc Packages
\PassOptionsToPackage{final}{graphicx}
\usepackage{%
	xcolor,%
	graphicx,%
	cancel,%
	booktabs,
	hyphenat,
	authoraftertitle,
	pdfpages
}

% Bibliography
\usepackage[
	backend=biber,
	backref=true,
	style=trad-abbrv,
	sorting=ynt
]{biblatex}
\addbibresource{C:/Users/Jhonny/Google-Drive/LaTeX/bib/general.bib}

% References
\usepackage{url} 
\usepackage{hyperref} 
\hypersetup{colorlinks=true,linkcolor=black,urlcolor=black}
\usepackage[spanish,nameinlink]{cleveref} 

% List Settings
\usepackage{enumitem} 
\setlist[enumerate]{label=\arabic*,left=-11pt}
\setlist[description]{font=\normalfont,leftmargin=\parindent}
\setlist[itemize]{label={\small\textbullet},left=-11pt}

% Margins: notes, figures, etc
\newcommand{\concretefont}{\sffamily} 
\usepackage{marginnote,sidenotes} 
\renewcommand*{\marginfont}{\small\concretefont}
\let\oldmarginpar\marginpar
\renewcommand{\marginpar}[1]{
	\oldmarginpar{\raggedright\small\concretefont #1}
}
\newcounter{nota}
\newcommand{\nota}[1]{\refstepcounter{nota}\textsuperscript{\thenota}%
	\marginpar{%
		\parbox{3.5cm}{\raggedright\thenota. #1}%
	}
}

% Custom \maketitle
\newcommand{\Asignatura}{}
\newcommand{\asignatura}[1]{\renewcommand{\Asignatura}{#1}}
% \setlength{\textparlen}{}
% \addtolength{\textparlen}{20pt}
\makeatletter
\def\@maketitle{%
  \newpage
  \null
  \let \footnote \thanks
  \begin{flushleft}\dispfont
	% \begin{minipage}{.25\textwidth}
	% \raggedleft\Asignatura
	% \end{minipage}
	% \enspace
	% \newsavebox{\tmpbox}\savebox{\tmpbox}{\parbox{.4\textwidth}{\raggedright\dispfont\@title}}%
	% \rule[-.9\ht\tmpbox]{1pt}{2\ht\tmpbox}\enspace%
	% \usebox{\tmpbox}%
	  {\addfontfeatures{LetterSpace=10}\MakeUppercase{\@title}\marginnote{\parbox{3cm}{\raggedright\@author}}\par\smallskip\titlerule}
	\medskip
	  {\small\raggedleft\Asignatura\ (0\the\month-\the\year)\par}
  	% \vskip 1em
  	% {}\par
	% {}\par
  \end{flushleft}
  % \marginpar{\@author\medskip}
  \vskip 1\baselineskip
}
\makeatother

% Caption setup
\usepackage{caption} 
\captionsetup{font={rm},justification=raggedright,singlelinecheck=false,skip=3pt}

% Section and subsection format
\usepackage[explicit]{titlesec}
\titleformat{\section}[hang]
{\raggedright\large\itshape}
	{\thesection}
	{1em}
	{#1}
	[]
\titlespacing*{\section}
	{0em}
	{1.5\baselineskip}
	{1.5\baselineskip}
\titleformat{\subsection}
{\flushleft\itshape}
	{\thesubsection}
	{.5em}
	{#1}
	[]
\titlespacing*{\subsection}
	{0em}
	{1\baselineskip}
	{1\baselineskip}
\titleformat{\paragraph}[runin]
	{\addfontfeatures{LetterSpace=10}\scshape}
	{}
	{0em}
	{#1}
	[.]
\titlespacing*{\paragraph}
	{0em}
	{0\parskip}
	{1\parskip}

% Table of contents setup
% \addto\captionsspanish{
% 	\renewcommand{\contentsname}{\concretefont Contenido}
% }
\let\oldtoc\tableofcontents
\renewcommand{\tableofcontents}{\marginpar{\bigskip\oldtoc}}
\usepackage{titletoc}
\titlecontents{section}
	[0em]
	{\vspace{-10pt}}
	{\contentsmargin{0pt}}
	{\contentsmargin{0pt}}
	{\contentspage}
	[\vspace{15pt}]
\titlecontents{subsection}
	[.5em]                              
	{\vspace{-11pt}}
	{\contentsmargin{0pt}\footnotesize}
	{\contentsmargin{0pt}\footnotesize}        
	{\contentspage}                 
	[\vspace{14pt}]

% Abstract redefine
	\renewenvironment{abstract}{
		\smallskip
		{\raggedright\large Resumen}\par\smallskip
	}{\par\medskip}
% Header and footer setup
\usepackage{fancyhdr}
\renewcommand{\headrulewidth}{0pt}
\setlength{\headheight}{14pt}
\pagestyle{fancy}
% \renewcommand{\sectionmark}[1]{\markright{#1}}
\fancyhf{}
\fancyhead[L]{\ifodd\value{page}\MyTitle\else\Asignatura\fi}
	\fancyhead[R]{\thepage}
\fancypagestyle{plain}{%
	\fancyhead[R]{}
	\fancyhead[L]{}
	\fancyfoot[R]{}%
	\fancyfoot[L]{}
	\fancyfoot[C]{}
}

% Theorem environments
\usepackage[thmmarks]{ntheorem}
	\theoremstyle{plain}
	\theoremindent0cm
	\theorempreskip{1\parskip}
	\theoremheaderfont{\normalfont}
	\theorembodyfont{\normalfont}
	\theoremseparator{.}
	\newtheorem{defi}{Definición}[section]
	\newtheorem{teo}{Teorema}[section]
	\newtheorem{cor}{Corolario}[teo]
	\newtheorem{prop}{Proposición}[section]
	\newtheorem{lem}{Lema}[section]
	\newtheorem{ejem}{Ejemplo}[section]
	\newtheorem{cejem}{Contraejemplo}[section]
	\newtheorem{ejer}{Ejercicio}[section]
	\theoremstyle{nonumberplain}
	\newtheorem{sol}{Solución}
	\newtheorem{obs}{Observación}
	\newtheorem{proof}{Demostración}
\renewcommand{\footnote}{\nota}
% \author{Jhonny Lanzuisi}


\title{Cuarto ejercicio,\\ espacios de hausdorff}
\author{Jhonny Lanzuisi, 15\,10759}
\asignatura{Topología 1}

\begin{document}
\maketitle
\tableofcontents
%\marginnote{
	%\begin{abstract}
		%Cuarto ejercico del curso de \AsignaturaN: Espacios de Hausdorff y
		%topología producto.
	%\end{abstract}
	%\tableofcontents
%}

\section[Enunciado]{Enunciado\nota{En \cite{munkres_topology_2014}: Capítulo 2, \S 19, ejercicio 3}}
Sean $A$ un cojunto de indices y $X_{\alpha} (\alpha\in A)$ una familia de espacio topológicos.
Demuestre que si los $X_{\alpha}$ son espacios de hausdorff
entonces el productoo
\[
	\prod_{\alpha\in A} X_{\alpha}
\]
es un espacio de Hausdorff tanto en la topología caja como en la topología producto.

\subsection{Solución}
Tomemos dos puntos $x,y$ distintos en $\prod X_{\alpha}$. Basta con construir
un entorno de $x$ que no contenga a $y$ (tanto en la topología caja como en la producto)
y el resultado buscado se obtendrá entonces haciendo un argumento simétrico para $y$.

Como $x$ y $y$ son distintos,
existe al menos un índice $\beta$ en $A$ tal que $x_{\beta}\neq y_{\beta}$.
Como $X_{\beta}$ es un espacio de Hausdorff, existe  un entorno $U$
(tanto en la topología producto como en la caja)
 en $X_{\beta}$ de $x_{\beta}$ que no intersectan a $y_{\beta}$.

Consideremos la famila de conjuntos $U_\alpha$ dada por
\[
	U_{\alpha}=
		\begin{cases}
			U &\text{si}\;\alpha=\beta,\\
			X_\alpha &\text{si}\;\alpha\neq \beta.
		\end{cases}
\]
Notemos que cada $U_\alpha$ es abierto en $X_\alpha$ y tomemos el producto
\[
	W=\prod_{\alpha\in A} U_\alpha.
\]
Evidentemente $W\subset\prod X_{\alpha}$. También, como $x_{\beta}\in U$ 
(por la forma en que se eligió $W$) y $x_\alpha\in X_{\alpha}$ para
$\alpha\neq\beta$, se sigue que $x\in W$. Por ser $W$ un producto
de cojuntos abiertos se sigue que es abierto en la topología
caja, como además todos menos una cantidad finita de los 
$W_\alpha$ son iguales a los $X_\alpha$ se tiene que $W$ también
es abierto en la topología producto.

Entonces, sin importar cual de las dos topologías tomemos (la caja o la producto)
el cojunto $W$ será un entorno del punto $x$. Solo queda por ver que este entorno
no intersecta al punto $y$. Esto último podemos verlo medianto un argumento por
contradicción.

Supongamos que $y\in W$. Entonces se tiene que $y_\alpha\in U_\alpha$ para cada $\alpha\in A$.
Pero esto implica, en particular, que $y_{\beta}\in U$.
Lo cual es una contradicción.

Hemos obtenido entonces que $W$ es un entorno de $x$ que no contiene a $y$.
De manera similar pude construirse un entorno $V$ de $y$ que no contenga a $x$
y queda demostrado que $\prod X_\alpha$ es un espacio de Hausdorff.
\printbibliography[
heading=bibintoc,
title={Referencias}
]
\end{document}
