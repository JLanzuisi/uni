\documentclass[fleqn,leqno,10pt,letterpaper,draft]{article}
% Misc TeX Settings
\hfuzz1pc
\overfullrule=2cm

% Language Setup
\usepackage{polyglossia}
\setmainlanguage[spanishoperators=all,]{spanish}
\PolyglossiaSetup{spanish}{,indentfirst=false}
\usepackage{csquotes} 

% Page dimensions
\usepackage[
	includehead,
	includefoot,
	top=1.5cm,
	bottom=1.5cm,
	left=2.5cm,
	right=8.5cm,
	marginparsep=0.5cm,
	marginparwidth=7cm,
]{geometry}

% Math
\usepackage{mathtools} 
\DeclareMathOperator{\Rea}{Re}
\DeclareMathOperator{\Ima}{Im}
\DeclareMathOperator{\car}{car}
\DeclareMathOperator{\traz}{tr}
\DeclareMathOperator{\gen}{gen}
\DeclareMathOperator{\mcm}{mcm}
\DeclareMathOperator{\id}{id}

% Text Fonts
\usepackage{unicode-math} 
\defaultfontfeatures{SmallCapsFeatures={LetterSpace=10},Scale=MatchLowercase,Numbers={OldStyle,Proportional}}
\setmainfont[
SmallCapsFont=* Caps,
]{Latin Modern Roman}
\setsansfont{Libertinus Sans}
\setmonofont[BoldFont=* Semibold]{Source Code Pro}
\newcommand{\dispfont}{} 
\frenchspacing
\setlength{\parindent}{1em}
\linespread{1.02}

% Microtype
\usepackage[
final,
% tracking=smallcaps,
% expansion=alltext,
% protrusion=true
]{microtype}
% \SetTracking{encoding=*,shape=sc}{50}

% Misc Packages
\PassOptionsToPackage{final}{graphicx}
\usepackage{%
	xcolor,%
	graphicx,%
	cancel,%
	booktabs,
	hyphenat,
	authoraftertitle,
	pdfpages
}

% Bibliography
\usepackage[
	backend=biber,
	backref=true,
	style=trad-abbrv,
	sorting=ynt
]{biblatex}
\addbibresource{C:/Users/Jhonny/Google-Drive/LaTeX/bib/general.bib}

% References
\usepackage{url} 
\usepackage{hyperref} 
\hypersetup{colorlinks=true,linkcolor=black,urlcolor=black}
\usepackage[spanish,nameinlink]{cleveref} 

% List Settings
\usepackage{enumitem} 
\setlist[enumerate]{label=\arabic*,left=-11pt}
\setlist[description]{font=\normalfont,leftmargin=\parindent}
\setlist[itemize]{label={\small\textbullet},left=-11pt}

% Margins: notes, figures, etc
\newcommand{\concretefont}{\sffamily} 
\usepackage{marginnote,sidenotes} 
\renewcommand*{\marginfont}{\small\concretefont}
\let\oldmarginpar\marginpar
\renewcommand{\marginpar}[1]{
	\oldmarginpar{\raggedright\small\concretefont #1}
}
\newcounter{nota}
\newcommand{\nota}[1]{\refstepcounter{nota}\textsuperscript{\thenota}%
	\marginpar{%
		\parbox{3.5cm}{\raggedright\thenota. #1}%
	}
}

% Custom \maketitle
\newcommand{\Asignatura}{}
\newcommand{\asignatura}[1]{\renewcommand{\Asignatura}{#1}}
% \setlength{\textparlen}{}
% \addtolength{\textparlen}{20pt}
\makeatletter
\def\@maketitle{%
  \newpage
  \null
  \let \footnote \thanks
  \begin{flushleft}\dispfont
	% \begin{minipage}{.25\textwidth}
	% \raggedleft\Asignatura
	% \end{minipage}
	% \enspace
	% \newsavebox{\tmpbox}\savebox{\tmpbox}{\parbox{.4\textwidth}{\raggedright\dispfont\@title}}%
	% \rule[-.9\ht\tmpbox]{1pt}{2\ht\tmpbox}\enspace%
	% \usebox{\tmpbox}%
	  {\addfontfeatures{LetterSpace=10}\MakeUppercase{\@title}\marginnote{\parbox{3cm}{\raggedright\@author}}\par\smallskip\titlerule}
	\medskip
	  {\small\raggedleft\Asignatura\ (0\the\month-\the\year)\par}
  	% \vskip 1em
  	% {}\par
	% {}\par
  \end{flushleft}
  % \marginpar{\@author\medskip}
  \vskip 1\baselineskip
}
\makeatother

% Caption setup
\usepackage{caption} 
\captionsetup{font={rm},justification=raggedright,singlelinecheck=false,skip=3pt}

% Section and subsection format
\usepackage[explicit]{titlesec}
\titleformat{\section}[hang]
{\raggedright\large\itshape}
	{\thesection}
	{1em}
	{#1}
	[]
\titlespacing*{\section}
	{0em}
	{1.5\baselineskip}
	{1.5\baselineskip}
\titleformat{\subsection}
{\flushleft\itshape}
	{\thesubsection}
	{.5em}
	{#1}
	[]
\titlespacing*{\subsection}
	{0em}
	{1\baselineskip}
	{1\baselineskip}
\titleformat{\paragraph}[runin]
	{\addfontfeatures{LetterSpace=10}\scshape}
	{}
	{0em}
	{#1}
	[.]
\titlespacing*{\paragraph}
	{0em}
	{0\parskip}
	{1\parskip}

% Table of contents setup
% \addto\captionsspanish{
% 	\renewcommand{\contentsname}{\concretefont Contenido}
% }
\let\oldtoc\tableofcontents
\renewcommand{\tableofcontents}{\marginpar{\bigskip\oldtoc}}
\usepackage{titletoc}
\titlecontents{section}
	[0em]
	{\vspace{-10pt}}
	{\contentsmargin{0pt}}
	{\contentsmargin{0pt}}
	{\contentspage}
	[\vspace{15pt}]
\titlecontents{subsection}
	[.5em]                              
	{\vspace{-11pt}}
	{\contentsmargin{0pt}\footnotesize}
	{\contentsmargin{0pt}\footnotesize}        
	{\contentspage}                 
	[\vspace{14pt}]

% Abstract redefine
	\renewenvironment{abstract}{
		\smallskip
		{\raggedright\large Resumen}\par\smallskip
	}{\par\medskip}
% Header and footer setup
\usepackage{fancyhdr}
\renewcommand{\headrulewidth}{0pt}
\setlength{\headheight}{14pt}
\pagestyle{fancy}
% \renewcommand{\sectionmark}[1]{\markright{#1}}
\fancyhf{}
\fancyhead[L]{\ifodd\value{page}\MyTitle\else\Asignatura\fi}
	\fancyhead[R]{\thepage}
\fancypagestyle{plain}{%
	\fancyhead[R]{}
	\fancyhead[L]{}
	\fancyfoot[R]{}%
	\fancyfoot[L]{}
	\fancyfoot[C]{}
}

% Theorem environments
\usepackage[thmmarks]{ntheorem}
	\theoremstyle{plain}
	\theoremindent0cm
	\theorempreskip{1\parskip}
	\theoremheaderfont{\normalfont}
	\theorembodyfont{\normalfont}
	\theoremseparator{.}
	\newtheorem{defi}{Definición}[section]
	\newtheorem{teo}{Teorema}[section]
	\newtheorem{cor}{Corolario}[teo]
	\newtheorem{prop}{Proposición}[section]
	\newtheorem{lem}{Lema}[section]
	\newtheorem{ejem}{Ejemplo}[section]
	\newtheorem{cejem}{Contraejemplo}[section]
	\newtheorem{ejer}{Ejercicio}[section]
	\theoremstyle{nonumberplain}
	\newtheorem{sol}{Solución}
	\newtheorem{obs}{Observación}
	\newtheorem{proof}{Demostración}
\renewcommand{\footnote}{\nota}
% \author{Jhonny Lanzuisi}


\title{Tercer Ejercicio}
\author{Jhonny Lanzuisi, 15\,10759}
\asignatura{Universidad Simon Bolivar, Topología 1}

\begin{document}
\maketitle
\tableofcontents
\marginpar{
	\begin{abstract}
		 \Asignatura, Tercera tarea. Espacios topológicos, bases, comparación de topologías y ordenes parciales. 
	\end{abstract}
}

\section[Enunciado]{Enunciado}

Sea $X$ un conjunto ordenado parcialmente. Sean $U_L(x)=\left\{ y\mid y\prec x \right\}$ y
$U_R(x)=\left\{ y\mid x\prec y \right\}$. \textit{Demuestre que:} 
\begin{enumerate}
	\item Las familias $\left\{ U_L(x) \right\}$, $\left\{ U_R(x) \right\}$ son bases de dos topologías
	$\mathcal{T}_L$ y $\mathcal{T}_R$, respectivamente, sobre $X$.
	\item $G$ esta en $\mathcal{T}_L$ si, y solo si, se cumple que 
		\[
			x\in G\implies U_L(x)\subset G.
		\]
	\item En $\mathcal{T}_L$ las intersecciones arbitrarias de conjuntos abiertos
		dan conjuntos abiertos.
	\item La topología discreta es la única mas fina que $\mathcal{T}_L$ y $\mathcal{T}_R$.
	\item Las topologías $\mathcal{T}_L$ y $\mathcal{T}_R$ no son comparables. 
\end{enumerate}

\section{Solución}

\paragraph{Parte 1}%

Primero veamos que las familias $\left\{ U_L(x) \right\}$, $\left\{ U_R(x) \right\}$
cubren el conjunto $X$. Esto no es muy complicado puesto que, para todo $x\in X$,
los conjuntos $ U_L(x) $ y $ U_R(x) $ contienen a $x$ 
(debido a la reflexividad de $\prec$) de donde se sigue que $X$ puede escribirse
como la unión de los $U_L(x)$ o como unión de los $U_R(x)$.

Tomemos ahora un elemento $y_1$ en $U_L(x_1)\cap U_L(x_2)$, donde
$x_1,x_2$ son elementos de $X$. Entonces
\[
	y_1\in U_L(y_1),
\]
por la misma razón que antes, y
\[
	U_L(y_1)\subset U_L(x_1)\cap U_L(x_2)
\]
puesto que si tomamos un $y\in U_L(y_1)$ entonces $y\prec y_1$, pero como
$y_1\prec x_1$ y $y_1\prec x_2$, se tiene que $y\prec x_1$ y $y\prec x_2$ por
transitividad. Hemos descubierto que para todo elemento en la intersección de dos $U_L$
podemos conseguir otro conjunto $U_L$ tal que contiene a dicho elemento y esta contenido en la
intersección, es decir, que la familia $\left\{ U_L(x) \right\}$ forma una base para una topología
sobre $X$ por el teorema~\ref{teo1}. Esta topología es $\mathcal{T}_L$.

Un argumento análogo al anterior nos da como resultado que la familia $\left\{ U_R(x) \right\}$ también es
base de una topología sobre $X$, y esta topología es $\mathcal{T}_R$.

\paragraph{Parte 2}
Supongamos que $G$ pertenece a $\mathcal{T}_L$, entonces $G$ se escribe como una unión arbitraria
de elementos básicos, es decir, 
\begin{align}
	G = \bigcup_\alpha U_L(x_\alpha).\label{eq1}
\end{align}
Si tomamos un $x\in G$ se sigue que $x$ debe pertenecer a alguno de los $U_L(x_\alpha)$. Como $x$ pertenece
a este elemento básico se tiene que $x\prec x_\alpha$ pero entonces, por definición de los $U_L$,
\[
	U_L(x)\subset U_L(x_\alpha)
\]
y por la igualdad en~\ref{eq1},
\[
	U_L(x)\subset U_L(x_\alpha)\subset G
\]
de donde se tiene, claramente, que $U_L(x)\subset G$.

Supongamos ahora que para cada $x\in G$ se cumple que
\[
	x\in G\implies U_L(x)\subset G.
\]
Pero como $U_L(x)$ es un elemento de la base de $\mathcal{T}_L$, la implicación anterior
da de forma inmediata, por el teorema~\ref{teo2}, que $G\in\mathcal{T}_L$.

\paragraph{Parte 3}%

Sea $\mathcal{A}=\left\{ A_\alpha\mid\alpha\in\mathscr{A} \right\}$ una familia de conjuntos abiertos
de $\mathcal{T}_L$, y consideremos la intersección
\[
	\bigcap\mathcal{A}.
\]

Si tomamos un $x\in\cap\mathcal{A}$ entonces $x$ pertenece a todos los $A_\alpha$. Como estos conjuntos
$A_\alpha$ son abiertos se sigue, por la parte anterior, que $U_L(x)$ esta contenido en todos
los $A_\alpha$. Pero esto es lo mismo que decir que
\[
	U_L(x)\subset\bigcap\mathcal{A},
\]
y, nuevamente por la parte anterior, se tiene que $\cap\mathcal{A}$ es un conjunto abierto. 

\paragraph{Parte 4}%

Sea $\mathcal{T}$ una topología sobre $X$ tal que
\begin{align}\label{eq2}
	\mathcal{T}_L\subset\mathcal{T}\quad\text{y}\quad\mathcal{T}_R\subset\mathcal{T}.
\end{align}
Sabemos que al menos una tal $\mathcal{T}$ existe: la topología discreta, a la que llamaremos $\mathcal{D}$, por lo que
tiene sentido preguntarse si existe otra topología con esta propiedad.

Veamos. Dado cualquier $x\in X$, la topología $\mathcal{T}$ debe cumplir (por~\ref{eq2})
\[
	U_L(x)\in\mathcal{T}\quad\text{y}\quad U_R(x)\in\mathcal{T}.
\]

Pero como $\mathcal{T}$ es una topología la intersección $U_L(x)\cap U_R(x)=\left\{ x \right\}$ debe pertenecer también a $\mathcal{T}.$
Entonces todos los conjuntos de la forma $\left\{ x \right\}$ ($x\in X$) pertenecen a $\mathcal{T}$, es decir, $\mathcal{D}\subset\mathcal{T}$.

Pero como $\mathcal{D}$ siempre es la topología mas fina, se tiene también $\mathcal{T}\subset\mathcal{D}$, por lo que
$\mathcal{D}=\mathcal{T}$.

Es decir, $\mathcal{D}$ es la única topología más fina que $\mathcal{T}_L$ y $\mathcal{T}_R$.
\paragraph{Parte 5}%

Notemos primero que se puede establecer un criterio análogo al de la parte 2
para caracterizar los conjuntos abiertos de $\mathcal{T}_R$, y la demostración de este hecho
es muy parecida a la de la parte 2.

Tomemos dos elementos $x_1,x_2$ de $X$ tales que $x_1\prec x_2$. Entonces el cojunto $U_L(x_1)$, que
es abierto en $\mathcal{T}_L$, \emph{no contiene} a ningun elemento $y$ que suceda a $x_1$ (por ejemplo,
no contiene a $x_2$) por lo que $U_R(x_1)\not\subset U_L(x_1)$ y por lo tanto $U_L(x_1)$ no es abierto
en $\mathcal{T}_R$. Un ejemplo análogo nos dará un conjunto abierto en $\mathcal{T}_R$ que no es abierto
en $\mathcal{T}_L$. Entonces se tiene que $\mathcal{T}_L\not\subset\mathcal{T}_R$ y $\mathcal{T}_R\not\subset\mathcal{T}_L$.

\section[Resultados Utilizados]{Resultados Utilizados}%
\label{sec:resultados_utilizados}

\begin{teo}\label{teo1}
	Sea $\mathcal{B}=\left\{ U_\alpha \mid \alpha\in \mathscr{M} \right\}$ una familia de subconjuntos de $X$
	que cubre a $X$ y satisface la siguiente condición:
	\begin{itemize}
		\item 	Para cada $\alpha,\beta\in\mathscr{M}\times\mathscr{M}$ y cada $x\in U_\alpha\cap U_\beta$,
	existe un $U_\gamma$ tal que $x\in U_\gamma\subset U_\alpha\cap U_\beta$.
	\end{itemize}
	
	Entonces el conjunto $\mathcal{T}(\mathcal{B})$ que consiste de $X,\emptyset$ y todas las uniones de miembros
	de $\mathcal{B}$ es una topología sobre $X$, es decir, $\mathcal{B}$ es la base de \emph{alguna}
	topología sobre $X$.
\end{teo}

\begin{teo}\label{teo2}
	Sea $\mathcal{B}\subset\mathcal{T}$ una base para la topología $\mathcal{T}$. Entonces un conjunto $A$
	es abierto (es decir, pertenece a $\mathcal{T}$) si, y solo si, para cada $x\in A$ existe un
	$U\in\mathcal{B}$ tal que $x\in U\subset A$.
\end{teo}

\printbibliography[
heading=bibintoc,
title={Referencias}
]
\end{document}
