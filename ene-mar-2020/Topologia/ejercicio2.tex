\documentclass[mid,fleqn,final,oneside]{tareas}
\clase{topología i}
\DeclareMathOperator{\id}{id}
\renewcommand{\theteo}{\arabic{teo}}
\begin{document}
\chapter{Segundo Ejercicio}
\reversemarginpar\identi{Jhonny Lanzuisi}{15\,10759}
\vspace{-.1em}
\section*{Enunciado\footnotemark}
\footnotetext{Ejercicio 4, pag. 109 del Munkres.}
Sean $X,Y$ dos conjuntos. Dados $x_0\in X$ y $y_0\in Y$, entonces las funciones $f:X\to X\times Y$ y 
$g: Y\to X\times Y$ definidas por
\[
	f(x)=(x,y_0)\quad\text{y}\quad g(y)=(x_0,y)
\]
son \emph{inclusiones}.

\section*{Solución}

La función $f$ es inyectiva puesto que
\[
f(x_1)=f(x_2)
\]
implica
\[
(x_1,y_0)=(x_2,y_0)
\]
y esto, por definición del producto cartesiano, implica
que $x_1=x_2$. Un razonamiento análogo nos dice que la
función $g$ también es inyectiva.

Entonces, si restringimos el rango de $f$ y $g$ a sus
imágenes directas obtenemos dos funciones $f',g'$
biyectivas. Para ver que $f,g$ son inclusiones solo
hace falta ver que $f',g'$ y sus inversas son
continuas.

Para ver que $f'$ es continua, notemos que esta función
se puede escribir de la forma
\[
f'=(\id_X,\gamma),
\]
donde $\id_X$ es la identidad de $X$ y $\gamma:X\to Y$ 
es la función constantemente igual a $y_0$.

Entonces la continuidad de $f'$ se sigue de la
continuidad de estas dos funciones\footnote{Véase el teorema 2\cite{cont:funccoord}}. Un argumento
análogo sirve para la función $g'$ y por lo tanto
$f',g'$ son ambas continuas.

Al igual que antes, una vez que veamos que $f'^{-1}$ es
continua un argumento completamente análogo nos dará
como resultado que $g'^{-1}$ es continua. Para ver la
continuidad de $f'^{-1}$ notemos que esta función esta
dada, al fijar un $y_0\in Y$, por
\[
f'^{-1}=\pi	
\]
donde $\pi$ es la proyección canónica de $X\times\left\{ y_0 \right\}$

sobre $X$. Entonces la continuidad de $f'^{-1}$ se
sigue de la continuidad de $\pi$\footnote{Véase el teorema~\ref{cont:proy}}.

Tenemos entonces que $f',g'$ y sus inversas son
continuas, por lo que $f,g$ son inclusiones.

\begin{obs}
	El hecho de que $f,g$ son inclusiones se puede ver de forma `geométrica' si imaginamos un plano en el que el conjunto $X$ y el cojunto $Y$ forman ejes perpendiculares. Entonces la función $f$ envía a $X$ en \emph{líneas horizontales} de `altura' $y_0$. Análogamente, $g$ envía a $Y$ en \emph{líneas verticales} cuya `distancia del orígen' es $x_0$.
\end{obs}
\section*{Resultados Utilizados}

\begin{teo}
	Las proyecciones sobre la primera y la segunda componente $\pi_1,\pi_2$ ---definidas en el producto cartesiano $X\times Y$--- son funciones continuas.	
\end{teo}

\begin{proof}[Grosso modo]
	Si $U$ es abierto entonces su imágen inversa $\pi_1^{-1}(U)$ es el conjunto $U\times Y$ que es abierto en $X\times Y$. De igual forma $\pi_2^{-1}(V)=X\times V$ es abierto.
\end{proof}

\begin{teo}
Sea $f:A\to X\times Y$ una función dada, para cada $a\in A$, por
\[
	f(a)=(f_1(a),f_2(a)).
\]
Entonces $f$ es \emph{continua} si, y solo si, las funciones
\[
	f_1:A\to X\quad\text{y}\quad f_2:A\to Y 
\]
son continuas. 
\end{teo}

\begin{proof}[Grosso modo]
	La demostración se basa en que $f$ se puede escribir como
	\[
		f_1(a)=\pi_1(f(a))\quad\text{y}\quad
		f_2(a)=\pi_2(f(a)).
	\]
	De donde la continuidad de $f$ implica que $f_1$ y $f_2$ son continuas
	al ser estas composición de funciones cotinuas, donde $\pi_1,\pi_2$ son las proyecciones en la primera y segunda componente.

	Por otro lado, si $U\times V$ es un elemento básico de la topología en $X\times Y$ entonces la identidad
	\[
		f^{-1}(U\times V) = f_1^{-1}(U)\cap f_2^{-1}(V)
	\]
	implica que $f$ es continua si $f_1,f_2$ lo son.
\end{proof}
\end{document}
