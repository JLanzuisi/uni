\documentclass[fleqn,leqno,11pt,letterpaper,final]{article}
% Misc TeX Settings
\hfuzz1pc
\overfullrule=2cm

% Language Setup
\usepackage{polyglossia}
\setmainlanguage[spanishoperators=all,]{spanish}
\PolyglossiaSetup{spanish}{,indentfirst=false}
\usepackage{csquotes} 

% Page dimensions
\usepackage[
	includehead,
	includefoot,
	top=1.5cm,
	bottom=1.5cm,
	left=2.5cm,
	right=8.5cm,
	marginparsep=0.5cm,
	marginparwidth=7cm,
]{geometry}

% Math
\usepackage{mathtools} 
\DeclareMathOperator{\Rea}{Re}
\DeclareMathOperator{\Ima}{Im}
\DeclareMathOperator{\car}{car}
\DeclareMathOperator{\traz}{tr}
\DeclareMathOperator{\gen}{gen}
\DeclareMathOperator{\mcm}{mcm}
\DeclareMathOperator{\id}{id}

% Text Fonts
\usepackage{unicode-math} 
\defaultfontfeatures{SmallCapsFeatures={LetterSpace=10},Scale=MatchLowercase,Numbers={OldStyle,Proportional}}
\setmainfont[
SmallCapsFont=* Caps,
]{Latin Modern Roman}
\setsansfont{Libertinus Sans}
\setmonofont[BoldFont=* Semibold]{Source Code Pro}
\newcommand{\dispfont}{} 
\frenchspacing
\setlength{\parindent}{1em}
\linespread{1.02}

% Microtype
\usepackage[
final,
% tracking=smallcaps,
% expansion=alltext,
% protrusion=true
]{microtype}
% \SetTracking{encoding=*,shape=sc}{50}

% Misc Packages
\PassOptionsToPackage{final}{graphicx}
\usepackage{%
	xcolor,%
	graphicx,%
	cancel,%
	booktabs,
	hyphenat,
	authoraftertitle,
	pdfpages
}

% Bibliography
\usepackage[
	backend=biber,
	backref=true,
	style=trad-abbrv,
	sorting=ynt
]{biblatex}
\addbibresource{C:/Users/Jhonny/Google-Drive/LaTeX/bib/general.bib}

% References
\usepackage{url} 
\usepackage{hyperref} 
\hypersetup{colorlinks=true,linkcolor=black,urlcolor=black}
\usepackage[spanish,nameinlink]{cleveref} 

% List Settings
\usepackage{enumitem} 
\setlist[enumerate]{label=\arabic*,left=-11pt}
\setlist[description]{font=\normalfont,leftmargin=\parindent}
\setlist[itemize]{label={\small\textbullet},left=-11pt}

% Margins: notes, figures, etc
\newcommand{\concretefont}{\sffamily} 
\usepackage{marginnote,sidenotes} 
\renewcommand*{\marginfont}{\small\concretefont}
\let\oldmarginpar\marginpar
\renewcommand{\marginpar}[1]{
	\oldmarginpar{\raggedright\small\concretefont #1}
}
\newcounter{nota}
\newcommand{\nota}[1]{\refstepcounter{nota}\textsuperscript{\thenota}%
	\marginpar{%
		\parbox{3.5cm}{\raggedright\thenota. #1}%
	}
}

% Custom \maketitle
\newcommand{\Asignatura}{}
\newcommand{\asignatura}[1]{\renewcommand{\Asignatura}{#1}}
% \setlength{\textparlen}{}
% \addtolength{\textparlen}{20pt}
\makeatletter
\def\@maketitle{%
  \newpage
  \null
  \let \footnote \thanks
  \begin{flushleft}\dispfont
	% \begin{minipage}{.25\textwidth}
	% \raggedleft\Asignatura
	% \end{minipage}
	% \enspace
	% \newsavebox{\tmpbox}\savebox{\tmpbox}{\parbox{.4\textwidth}{\raggedright\dispfont\@title}}%
	% \rule[-.9\ht\tmpbox]{1pt}{2\ht\tmpbox}\enspace%
	% \usebox{\tmpbox}%
	  {\addfontfeatures{LetterSpace=10}\MakeUppercase{\@title}\marginnote{\parbox{3cm}{\raggedright\@author}}\par\smallskip\titlerule}
	\medskip
	  {\small\raggedleft\Asignatura\ (0\the\month-\the\year)\par}
  	% \vskip 1em
  	% {}\par
	% {}\par
  \end{flushleft}
  % \marginpar{\@author\medskip}
  \vskip 1\baselineskip
}
\makeatother

% Caption setup
\usepackage{caption} 
\captionsetup{font={rm},justification=raggedright,singlelinecheck=false,skip=3pt}

% Section and subsection format
\usepackage[explicit]{titlesec}
\titleformat{\section}[hang]
{\raggedright\large\itshape}
	{\thesection}
	{1em}
	{#1}
	[]
\titlespacing*{\section}
	{0em}
	{1.5\baselineskip}
	{1.5\baselineskip}
\titleformat{\subsection}
{\flushleft\itshape}
	{\thesubsection}
	{.5em}
	{#1}
	[]
\titlespacing*{\subsection}
	{0em}
	{1\baselineskip}
	{1\baselineskip}
\titleformat{\paragraph}[runin]
	{\addfontfeatures{LetterSpace=10}\scshape}
	{}
	{0em}
	{#1}
	[.]
\titlespacing*{\paragraph}
	{0em}
	{0\parskip}
	{1\parskip}

% Table of contents setup
% \addto\captionsspanish{
% 	\renewcommand{\contentsname}{\concretefont Contenido}
% }
\let\oldtoc\tableofcontents
\renewcommand{\tableofcontents}{\marginpar{\bigskip\oldtoc}}
\usepackage{titletoc}
\titlecontents{section}
	[0em]
	{\vspace{-10pt}}
	{\contentsmargin{0pt}}
	{\contentsmargin{0pt}}
	{\contentspage}
	[\vspace{15pt}]
\titlecontents{subsection}
	[.5em]                              
	{\vspace{-11pt}}
	{\contentsmargin{0pt}\footnotesize}
	{\contentsmargin{0pt}\footnotesize}        
	{\contentspage}                 
	[\vspace{14pt}]

% Abstract redefine
	\renewenvironment{abstract}{
		\smallskip
		{\raggedright\large Resumen}\par\smallskip
	}{\par\medskip}
% Header and footer setup
\usepackage{fancyhdr}
\renewcommand{\headrulewidth}{0pt}
\setlength{\headheight}{14pt}
\pagestyle{fancy}
% \renewcommand{\sectionmark}[1]{\markright{#1}}
\fancyhf{}
\fancyhead[L]{\ifodd\value{page}\MyTitle\else\Asignatura\fi}
	\fancyhead[R]{\thepage}
\fancypagestyle{plain}{%
	\fancyhead[R]{}
	\fancyhead[L]{}
	\fancyfoot[R]{}%
	\fancyfoot[L]{}
	\fancyfoot[C]{}
}

% Theorem environments
\usepackage[thmmarks]{ntheorem}
	\theoremstyle{plain}
	\theoremindent0cm
	\theorempreskip{1\parskip}
	\theoremheaderfont{\normalfont}
	\theorembodyfont{\normalfont}
	\theoremseparator{.}
	\newtheorem{defi}{Definición}[section]
	\newtheorem{teo}{Teorema}[section]
	\newtheorem{cor}{Corolario}[teo]
	\newtheorem{prop}{Proposición}[section]
	\newtheorem{lem}{Lema}[section]
	\newtheorem{ejem}{Ejemplo}[section]
	\newtheorem{cejem}{Contraejemplo}[section]
	\newtheorem{ejer}{Ejercicio}[section]
	\theoremstyle{nonumberplain}
	\newtheorem{sol}{Solución}
	\newtheorem{obs}{Observación}
	\newtheorem{proof}{Demostración}
\renewcommand{\footnote}{\nota}
% \author{Jhonny Lanzuisi}


\title{ejercicios de topología}
\author{Jhonny Lanzuisi, 15\,10759\\\url{jalb97@gmail.com}}
\asignatura{Topología I}

\begin{document}
\maketitle
\tableofcontents
%\marginnote{
	%\begin{abstract}
		%Ejercicios del curso \AsignaturaN: Espacios topológicos, funciones continuas e inclusiones, espacios de Hausdorff, topología producto y espacios conexos.
	%\end{abstract}
	%\tableofcontents
%}

\section{Ejercicio 1: Enunciado}
\hspace{-.6em}\footnotemark Sea $X$ un conjunto y sea $\mathcal{T}_c$ el la
colección de todo los conjuntos $U$ de $X$ tales que
$X\setminus U$ es numerable o es todo $X$. Demuestre
que $\mathcal{T}_c$ es una topología sobre $X$.
\footnotetext{En \cite{munkres_topology_2014}: \S13, Ejercicio 3}

¿Será la colección
\[
\mathcal{T}_\infty = \left\{ U \mid X\setminus U\;
\text{es infinito o es vacío o es todo $X$} \right\}
\]
una topología sobre $X$?

\subsection{Solución}

Primero que todo, el conjunto vacío pertenece a $
\mathcal{T}_c $ debido a que $ X\setminus\emptyset=X
$ y $X\in \mathcal{T}_c$ por la definición de
$\mathcal{T}_c$.  También se
tiene que $ X $ es un conjunto abierto puesto que $
X\setminus X=\emptyset $ y el conjunto vacío es finito.

Supongamos que $
\left\{ U_k \right\}$ es una familia de elementos de $
\mathcal{T}_c $.  Entonces%\footnote{Este tipo de igualdades se
	%siguen de las leyes de De Morgan}
\[
X\setminus\bigcup U_k= \bigcap (X\setminus U_k).
\]
Pero el lado derecho de la igualdad es numerable\footnote{Véase el Corolario~\ref{num:inter}} puesto
que estas intersecciones son subconjuntos de todos los
$ X\setminus U_k $ y estos últimos son numerables.

Supongamos ahora que $ \left\{ U_1,\dots,U_n \right\} $
son una cantidad finita de elementos de $
\mathcal{T}_c. $ Entonces
\[
X\setminus\bigcap_{i=1}^{n}U_i =
\bigcup_{i=1}^{n} X\setminus U_i.
\]
Donde el lado derecho de la igualdad es numerable pues
la unión de conjuntos numerables es numerable\footnote{Véase el teorema~\ref{num:union}}.

Hemos visto que las uniones arbitrarias y las
intersecciones finitas de elementos de $\mathcal{T}_c$
pertenecen nuevamente a $\mathcal{T}_c$, esto es, que
$\mathcal{T}_c$ es una topología sobre $X$.

En el caso del conjunto $ \mathcal{T}_\infty $ se tiene
que \emph{solo es} una topología sobre $ X $ cuando
$X$ es finito, en cuyo caso $\mathcal{T}_\infty$
coindice con la topología indiscreta.
Uno puede ver fácilmente que existen conjuntos $X$
para los cuales $\mathcal{T}_\infty$ no es una
topología: pensemos por ejemplo en el caso
$X=\mathbb{Z}$ y los subconjuntos de los enteros
positivos y los enteros negativos, ambos sin incluir al
cero, cuya unión no pertenece a $\mathcal{T}_\infty$.

Pero podemos decir, como señalábamos antes, bastante
más: $\mathcal{T}_\infty$ es una topología si y sólo si $X$ es finito, en cuyo caso
$\mathcal{T}_\infty = \{\emptyset, X\}$, la topología indiscreta. 

Si $X$ es finito entonces cada $A\subset X$ con $A\not = \emptyset,X$
tiene complemento finito y por lo tanto $A$ no está en  $\mathcal{T}_\infty$.
Veamos ahora que si $X$ es infinito entonces $\mathcal{T}_\infty$ no es una topología.
Sea $(x_n\colon n\in\mathbb{N})$ una sucesión infinita de elementos distintos dos a dos de $X$, 
la cual podemos elegir inductivamente, ya que  $X \setminus \{x_0, x_1, \ldots,x_n\}$ es
no vacío para cada $n$. 
Si $U = X\setminus \{x_{2n}\}_{n\geq 0}$ y $V = \cup X \setminus (\{x_0\} \cup \{x_{2n+1}\}_{n\geq 0})$  entonces
ambos $U$ y $V$ están en  $\mathcal{T}_\infty$ pero $U\cup V = X \setminus \{x_0\}$ no está
en    $\mathcal{T}_\infty$.

\section{Ejercicio 2: Enunciado}
\hspace{-.6em}\footnotemark Sean $X,Y$ dos conjuntos. Dados $x_0\in X$ y $y_0\in Y$, entonces las funciones $f:X\to X\times Y$ y
\footnotetext{En \cite{munkres_topology_2014}: \S18, Ejercicio 4}
$g: Y\to X\times Y$ definidas por
\[
f(x)=(x,y_0)\quad\text{y}\quad g(y)=(x_0,y)
\]
son \emph{inclusiones}.

\subsection{Solución}

La función $f$ es inyectiva puesto que
\[
f(x_1)=f(x_2)
\]
implica
\[
(x_1,y_0)=(x_2,y_0)
\]
y esto, por definición del producto cartesiano, implica
que $x_1=x_2$. Un razonamiento análogo nos dice que la
función $g$ también es inyectiva.

Entonces, si restringimos el rango de $f$ y $g$ a sus
imágenes directas obtenemos dos funciones $f',g'$
biyectivas. Para ver que $f,g$ son inclusiones solo
hace falta ver que $f',g'$ y sus inversas son
continuas.

Para ver que $f'$ es continua, notemos que esta función
se puede escribir de la forma
\[
f'=(\id_X,\gamma),
\]
donde $\id_X$ es la identidad de $X$ y $\gamma:X\to Y$ 
es la función constantemente igual a $y_0$.

Entonces la continuidad de $f'$ se sigue de la
continuidad de estas dos funciones\footnote{Véase el teorema 2\ref{cont:funccoord}}. Un argumento
análogo sirve para la función $g'$ y por lo tanto
$f',g'$ son ambas continuas.

Al igual que antes, una vez que veamos que $f'^{-1}$ es
continua un argumento completamente análogo nos dará
como resultado que $g'^{-1}$ es continua. Para ver la
continuidad de $f'^{-1}$ notemos que esta función esta
dada, al fijar un $y_0\in Y$, por
\[
f'^{-1}=\pi	
\]
donde $\pi$ es la proyección canónica de $X\times\left\{ y_0 \right\}$
sobre $X$. Entonces la continuidad de $f'^{-1}$ se
sigue de la continuidad de $\pi$\footnote{Véase el teorema~\ref{cont:proy}}.

Tenemos entonces que $f',g'$ y sus inversas son
continuas, por lo que $f,g$ son inclusiones.

\begin{obs}
	El hecho de que $f,g$ son inclusiones se puede ver de forma `geométrica' si imaginamos un plano en el que el conjunto $X$ y el cojunto $Y$ forman ejes perpendiculares. Entonces la función $f$ envía a $X$ en \emph{líneas horizontales}. Análogamente, $g$ envía a $Y$ en \emph{líneas verticales}.
\end{obs}

\section{Ejercicio 3: Enunciado}
\hspace{-.6em}\footnotemark Sea $X$ un conjunto ordenado parcialmente. Sean $U_L(x)=\left\{ y\mid y\prec x \right\}$ y
$U_R(x)=\left\{ y\mid x\prec y \right\}$. Demuestre que:
\footnotetext{Dugundji\cite{dugundji_topology_1987}. Problemas del capítulo 3, sección 3, Nº6}
\begin{enumerate}
	\item Las familias $\left\{ U_L(x) \right\}$, $\left\{ U_R(x) \right\}$ son bases de dos topologías
	$\mathcal{T}_L$ y $\mathcal{T}_R$, respectivamente, sobre $X$.
	\item $G$ esta en $\mathcal{T}_L$ si, y solo si, se cumple que 
	\[
	x\in G\implies U_L(x)\subset G.
	\]
	\item En $\mathcal{T}_L$ las intersecciones arbitrarias de conjuntos abiertos
	dan conjuntos abiertos.
	\item La topología discreta es la única mas fina que $\mathcal{T}_L$ y $\mathcal{T}_R$.
	\item Las topologías $\mathcal{T}_L$ y $\mathcal{T}_R$ no son comparables. 
\end{enumerate}

\subsection{Solución}

\paragraph{Parte 1}%

Primero veamos que las familias $\left\{ U_L(x) \right\}$, $\left\{ U_R(x) \right\}$
cubren el conjunto $X$. Esto no es muy complicado puesto que, para todo $x\in X$,
los conjuntos $ U_L(x) $ y $ U_R(x) $ contienen a $x$ 
(debido a la reflexividad de $\prec$) de donde se sigue que $X$ puede escribirse
como la unión de los $U_L(x)$ o como unión de los $U_R(x)$.

Tomemos ahora un elemento $y_1$ en $U_L(x_1)\cap U_L(x_2)$, donde
$x_1,x_2$ son elementos de $X$. Entonces
\[
y_1\in U_L(y_1),
\]
por la misma razón que antes, y
\[
U_L(y_1)\subset U_L(x_1)\cap U_L(x_2)
\]
puesto que si tomamos un $y\in U_L(y_1)$ entonces $y\prec y_1$, pero como
$y_1\prec x_1$ y $y_1\prec x_2$, se tiene que $y\prec x_1$ y $y\prec x_2$ por
transitividad. Hemos descubierto que para todo elemento en la intersección de dos $U_L$
podemos conseguir otro conjunto $U_L$ tal que contiene a dicho elemento y esta contenido en la
intersección, es decir, que la familia $\left\{ U_L(x) \right\}$ forma una base para una topología
sobre $X$ por el teorema~\ref{bases}. Esta topología es $\mathcal{T}_L$.

Un argumento análogo al anterior nos da como resultado que la familia $\left\{ U_R(x) \right\}$ también es
base de una topología sobre $X$, y esta topología es $\mathcal{T}_R$.

\paragraph{Parte 2}
Supongamos que $G$ pertenece a $\mathcal{T}_L$, entonces $G$ se escribe como una unión arbitraria
de elementos básicos, es decir, 
\begin{align}
G = \bigcup_\alpha U_L(x_\alpha).\label{eq1}
\end{align}
Si tomamos un $x\in G$ se sigue que $x$ debe pertenecer a alguno de los $U_L(x_\alpha)$. Como $x$ pertenece
a este elemento básico se tiene que $x\prec x_\alpha$ pero entonces, por definición de los $U_L$,
\[
U_L(x)\subset U_L(x_\alpha)
\]
y por la igualdad en~\ref{eq1},
\[
U_L(x)\subset U_L(x_\alpha)\subset G
\]
de donde se tiene, claramente, que $U_L(x)\subset G$.

Supongamos ahora que para cada $x\in G$ se cumple que
\[
x\in G\implies U_L(x)\subset G.
\]
Pero como $U_L(x)$ es un elemento de la base de $\mathcal{T}_L$, la implicación anterior
da de forma inmediata, por el teorema~\ref{abiertos}, que $G\in\mathcal{T}_L$.

\paragraph{Parte 3}%

Sea $\mathcal{A}=\left\{ A_\alpha\mid\alpha\in\mathscr{A} \right\}$ una familia de conjuntos abiertos
de $\mathcal{T}_L$, y consideremos la intersección
\[
\bigcap\mathcal{A}.
\]

Si tomamos un $x\in\cap\mathcal{A}$ entonces $x$ pertenece a todos los $A_\alpha$. Como estos conjuntos
$A_\alpha$ son abiertos se sigue, por la parte anterior, que $U_L(x)$ esta contenido en todos
los $A_\alpha$. Pero esto es lo mismo que decir que
\[
U_L(x)\subset\bigcap\mathcal{A},
\]
y, nuevamente por la parte anterior, se tiene que $\cap\mathcal{A}$ es un conjunto abierto. 

\paragraph{Parte 4}%

Sea $\mathcal{T}$ una topología sobre $X$ tal que
\begin{align}\label{eq2}
\mathcal{T}_L\subset\mathcal{T}\quad\text{y}\quad\mathcal{T}_R\subset\mathcal{T}.
\end{align}
Sabemos que al menos una tal $\mathcal{T}$ existe: la topología discreta, a la que llamaremos $\mathcal{D}$, por lo que
tiene sentido preguntarse si existe otra topología con esta propiedad.

Veamos. Dado cualquier $x\in X$, la topología $\mathcal{T}$ debe cumplir (por~\ref{eq2})
\[
U_L(x)\in\mathcal{T}\quad\text{y}\quad U_R(x)\in\mathcal{T}.
\]

Pero como $\mathcal{T}$ es una topología la intersección $U_L(x)\cap U_R(x)=\left\{ x \right\}$ debe pertenecer también a $\mathcal{T}.$
Entonces todos los conjuntos de la forma $\left\{ x \right\}$ ($x\in X$) pertenecen a $\mathcal{T}$, es decir, $\mathcal{D}\subset\mathcal{T}$.

Pero como $\mathcal{D}$ siempre es la topología mas fina, se tiene también $\mathcal{T}\subset\mathcal{D}$, por lo que
$\mathcal{D}=\mathcal{T}$.

Es decir, $\mathcal{D}$ es la única topología más fina que $\mathcal{T}_L$ y $\mathcal{T}_R$.
\paragraph{Parte 5}%

Notemos primero que se puede establecer un criterio análogo al de la parte 2
para caracterizar los conjuntos abiertos de $\mathcal{T}_R$, y la demostración de este hecho
es muy parecida a la de la parte 2.

Tomemos dos elementos $x_1,x_2$ de $X$ tales que $x_1\prec x_2$. Entonces el cojunto $U_L(x_1)$, que
es abierto en $\mathcal{T}_L$, \emph{no contiene} a ningun elemento $y$ que suceda a $x_1$ (por ejemplo,
no contiene a $x_2$) por lo que $U_R(x_1)\not\subset U_L(x_1)$ y por lo tanto $U_L(x_1)$ no es abierto
en $\mathcal{T}_R$. Un ejemplo análogo nos dará un conjunto abierto en $\mathcal{T}_R$ que no es abierto
en $\mathcal{T}_L$. Entonces se tiene que $\mathcal{T}_L\not\subset\mathcal{T}_R$ y $\mathcal{T}_R\not\subset\mathcal{T}_L$.

\section{Ejercicio 4: Enunciado}
\hspace{-.6em}\footnotemark 
Sean $A$ un cojunto de indices y $X_{\alpha} (\alpha\in A)$ una familia de espacio topológicos.
Demuestre que si los $X_{\alpha}$ son espacios de hausdorff
entonces el producto
\[
\prod_{\alpha\in A} X_{\alpha}
\]
es un espacio de Hausdorff tanto en la topología caja como en la topología producto.
\footnotetext{En \cite{munkres_topology_2014}: \S 19, ejercicio 3}

\subsection{Solución}
Tomemos dos puntos $x,y$ distintos en $\prod X_{\alpha}$. Basta con construir
un entorno de $x$ que no contenga a $y$ (tanto en la topología caja como en la producto)
y el resultado buscado se obtendrá entonces haciendo un argumento simétrico para $y$.

Como $x$ y $y$ son distintos,
existe al menos un índice $\beta$ en $A$ tal que $x_{\beta}\neq y_{\beta}$.
Como $X_{\beta}$ es un espacio de Hausdorff, existe  un entorno $U$
(tanto en la topología producto como en la caja)
en $X_{\beta}$ de $x_{\beta}$ que no intersectan a $y_{\beta}$.

Consideremos la famila de conjuntos $U_\alpha$ dada por
\[
U_{\alpha}=
\begin{cases}
U &\text{si}\;\alpha=\beta,\\
X_\alpha &\text{si}\;\alpha\neq \beta.
\end{cases}
\]
Notemos que cada $U_\alpha$ es abierto en $X_\alpha$ y tomemos el producto
\[
W=\prod_{\alpha\in A} U_\alpha.
\]
Evidentemente $W\subset\prod X_{\alpha}$. También, como $x_{\beta}\in U$ 
(por la forma en que se eligió $W$) y $x_\alpha\in X_{\alpha}$ para
$\alpha\neq\beta$, se sigue que $x\in W$. Por ser $W$ un producto
de cojuntos abiertos se sigue que es abierto en la topología
caja, como además todos menos una cantidad finita de los 
$W_\alpha$ son iguales a los $X_\alpha$ se tiene que $W$ también
es abierto en la topología producto.

Entonces, sin importar cual de las dos topologías tomemos (la caja o la producto)
el cojunto $W$ será un entorno del punto $x$. Solo queda por ver que este entorno
no intersecta al punto $y$. Esto último podemos verlo medianto un argumento por
contradicción.

Supongamos que $y\in W$. Entonces se tiene que $y_\alpha\in U_\alpha$ para cada $\alpha\in A$.
Pero esto implica, en particular, que $y_{\beta}\in U$.
Lo cual es una contradicción.

Hemos obtenido entonces que $W$ es un entorno de $x$ que no contiene a $y$.
De manera similar pude construirse un entorno $V$ de $y$ que no contenga a $x$
y queda demostrado que $\prod X_\alpha$ es un espacio de Hausdorff.
\section{Ejercicio 5: Enunciado}

\hspace{-.6em}\footnotemark Sea $X_{\alpha\in J}$ una familia indexada de espacios conexos. Sea
\[
	X=\prod_{\alpha\in J} X_\alpha,
\]
y sea $a=a_\alpha$ un punto fijo de $X$.
\footnotetext{En \cite{munkres_topology_2014}: \S23, Ejercicio 10}
\begin{enumerate}
	\item Dado un subcojunto $K$ de $J$ finito, llamemos $X_K$ al subespacio de $X$ dado por los puntos
		$x$ tales que $x_\alpha=a_\alpha$ si $\alpha\not\in K$. Demuestre que $X_K$ es conexo.
	\item Demuestre que la unión $Y$ de todos los $X_K$ es conexa.
	\item Demuestre que $Y$ es la clausura de $X$ y concluya que $X$ es conexo.
\end{enumerate}

\subsection{Solución}
\paragraph{Primera Parte}
Sean $\alpha_1,\dots,\alpha_n$ los elementos de $K$, llamemos $P$ al producto $\prod_{\alpha\in K}X_\alpha$,
y consideremos la función 
$f:P\to X_K$ dada, para $x\in P$, por $f(x)=z$ donde
\[
	z_\alpha= \begin{cases}
		x_\alpha &\text{si}\; \alpha\in K,\\
		a_\alpha &\text{si}\; \alpha\in J-K.
	\end{cases}
\]
Entonces $P$ y $X_K$ son homeomorfos a traves de esta $f$. Como $P$ es conexo por
el teorema~\ref{conex:prod}, se tiene que $X_K$ es conexo por el teorema~\ref{conex:contfunct}.

\paragraph{segunda parte}
Sea $Y$ la union de los $X_K$. Como los elementos de cualquier $X_K$ coninciden con
$a$ en las coordenadas que pertenecen a $K$, se sigue que el punto $a$ esta en
todos los $X_K$ de forma casi inmediata. La conexidad de $Y$ es entonces consecuencia del
teorema~\ref{conex:union}.

\paragraph{tercera parte}
Elijamos un punto $(x_\alpha)\in X$. Queremos ver que todo entorno de $(x_\alpha)$ contiene
un punto de $Y$. Elijamos un entorno $U$ de $(x_\alpha)$. Como estamos bajo
la topología producto, este entorno $U$ es un producto de abiertos donde todos
menos una cantidad finita de ellos son iguales a los $X_\alpha$. Sea
$K=\{\alpha_1,\dots,\alpha_n\}$ esta cantidad finita de índices.

Entonces podemos hayar un punto $(y_\alpha)$ que esta en $X_K$ y  en $U$ de la siguiente manera:
hagamos $y_\alpha=a_\alpha$ para $\alpha\notin K$ y para los $\alpha\in K$ tomamos $y_\alpha$
tales que pertenecen a $U_\alpha$, donde los $U_\alpha$ son los términos
del producto, que conforma a $U$, que difieren de los $X_\alpha$.

Luego como $(y_\alpha)\in X_K$ entonces también pertenece a $Y$ y se tiene
que $X$ es la clausura de $Y$.

Por lo que $X$ es conexo por el teorema~\ref{conex:cerra}. Todo lo anterior en realidad ha demostrado
el siguiente teorema.
\begin{teo}
	En la topología producto, el producto arbitrario de espacios conexos es conexo.	
\end{teo}

\section{Resultados Utilizados}
\begin{cor}\label{num:inter}
	Un subconjunto de un conjunto numerable es numerable
\end{cor}
\begin{teo}\label{num:union}
	La unión numerable de conjuntos numerables es numerable.
\end{teo}
\begin{teo}\label{cont:proy}
	Las proyecciones sobre la primera y la segunda componente $\pi_1,\pi_2$ ---definidas en el producto cartesiano $X\times Y$--- son funciones continuas.	
\end{teo}

\begin{proof}[Grosso modo]
	Si $U$ es abierto entonces su imágen inversa $\pi_1^{-1}(U)$ es el conjunto $U\times Y$ que es abierto en $X\times Y$. De igual forma $\pi_2^{-1}(V)=X\times V$ es abierto.
\end{proof}

\begin{teo}\label{cont:funccoord}
	Sea $f:A\to X\times Y$ una función dada, para cada $a\in A$, por
	\[
	f(a)=(f_1(a),f_2(a)).
	\]
	Entonces $f$ es \emph{continua} si, y solo si, las funciones
	\[
	f_1:A\to X\quad\text{y}\quad f_2:A\to Y 
	\]
	son continuas. 
\end{teo}

\begin{proof}[Grosso modo]
	La demostración se basa en que $f$ se puede escribir como
	\[
	f_1(a)=\pi_1(f(a))\quad\text{y}\quad
	f_2(a)=\pi_2(f(a)).
	\]
	De donde la continuidad de $f$ implica que $f_1$ y $f_2$ son continuas
	al ser estas composición de funciones cotinuas, donde $\pi_1,\pi_2$ son las proyecciones en la primera y segunda componente.
	
	Por otro lado, si $U\times V$ es un elemento básico de la topología en $X\times Y$ entonces la identidad
	\[
	f^{-1}(U\times V) = f_1^{-1}(U)\cap f_2^{-1}(V)
	\]
	implica que $f$ es continua si $f_1,f_2$ lo son.
\end{proof}

\begin{teo}\label{bases}
	Sea $\mathcal{B}=\left\{ U_\alpha \mid \alpha\in \mathscr{M} \right\}$ una familia de subconjuntos de $X$
	que cubre a $X$ y satisface la siguiente condición:
	\begin{itemize}
		\item 	Para cada $\alpha,\beta\in\mathscr{M}\times\mathscr{M}$ y cada $x\in U_\alpha\cap U_\beta$,
		existe un $U_\gamma$ tal que $x\in U_\gamma\subset U_\alpha\cap U_\beta$.
	\end{itemize}
	
	Entonces el conjunto $\mathcal{T}(\mathcal{B})$ que consiste de $X,\emptyset$ y todas las uniones de miembros
	de $\mathcal{B}$ es una topología sobre $X$, es decir, $\mathcal{B}$ es la base de \emph{alguna}
	topología sobre $X$\footnote{En ~\cite{dugundji_topology_1987}, Capítulo 3}.
\end{teo}

\begin{teo}\label{abiertos}
	Sea $\mathcal{B}\subset\mathcal{T}$ una base para la topología $\mathcal{T}$. Entonces un conjunto $A$
	es abierto (es decir, pertenece a $\mathcal{T}$) si, y solo si, para cada $x\in A$ existe un
	$U\in\mathcal{B}$ tal que $x\in U\subset A$\footnote{En ~\cite{dugundji_topology_1987}, Capítulo 3}.
\end{teo}
\begin{teo}\label{conex:contfunct}
	La imagen de un espacio conexo bajo una función continua es conexo
\end{teo}
\begin{teo}\label{conex:prod}
	El producto finito de espacios conexos es conexo.
\end{teo}
\begin{teo}\label{conex:union}
	La unión de una colección de subespacios conexos de un espacio $X$ que tienen un
	punto en común es conexa.
\end{teo}
\begin{teo}\label{conex:cerra}
	Sea $A$ un subespacio conexo de $X$. Si $A\subset B\subset\bar A$, entonces $B$
	también es conexo.
\end{teo}
\begin{flushright}\scriptsize
	(Todos los resultados (excepto los teoremas \ref{bases},\ref{abiertos}) fueron tomados de \cite{munkres_topology_2014})
\end{flushright}
\printbibliography[
heading=bibintoc,
title={Referencias}
]
\end{document}
