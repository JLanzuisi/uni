\chapter{Sucesiones}

Una \Keyword{sucesión} de números reales es una función cuyo dominio son los naturales,
y cuyo rango es un subconjunto de los reales. Si el rango es un subconjunto acotado
de $\mathbb{R}$ diremos que la suseción esta \emph{acotada}.

\begin{defi}
\normalmarginpar\marginnote{Definición de convergencia.}
	Una sucesión $\left\{ a_{n} \right\}$ de números reales \Keyword{converge} a un \emph{límite} $L$ si se cumple que:

	Dado $\epsilon>0$ existe un $N\in\mathbb{N}$ tal que, para todo $n>N$,
	\[
		\left\lvert a_{n}-L \right\rvert<\epsilon.
	\]
\end{defi}

\begin{ejem} Esta es una lista de varias sucesiones, algunas convergentes, otras divergentes.
	\begin{itemize}
		\item $\left\{ a_{n} \right\}=1/n$ converge a $0$ y esta acotada.
		\item $\left\{ a_{n} \right\}=n^2$ no esta acotada y diverge.
		\item $\left\{ a_{n} \right\}=(-1^n)$ esta acotada, el rango es finito, y diverge.
		\item $\left\{ a_{n} \right\}=1$ esta acotada, el rango es finito, y converge.
	\end{itemize}	
\end{ejem}

El siguiente teorema da varias propiedades sobre las sucesiones convergentes.

\begin{teo}
	Sean $\left\{ a_{n} \right\}$ una suseción de número reales. Entonces se cumple que
	\begin{enumerate}
		\item $\left\{ a_{n} \right\}$ converge a $p$ si, y solo si,
			$p$ contienen todos salvo una cantidad finita de puntos de $\left\{ a_{n} \right\}$.
			Esto es, $p$ es un punto de acumulación.
		\item Si $\left\{ a_{n} \right\}$ converge a dos límites $p_1,p_2$ entonces $p_1=p_{2}$.
		\item Si $\left\{ a_{n} \right\}$ converge entonces esta acotada.
	\end{enumerate}
\end{teo}

El siguiente teomera nos dice como se comporta el álgebra de sucesiones.

\begin{teo}
	Supongamos que $\{ a_n \}$ y $\{ b_n \}$ son dos sucesiones convergentes a $a,b$ respectivamente. Entonces
	\begin{itemize}
		\item $ \lim_{n \to \infty} (\{ a_n \}+\{ b_n \}) = a+b $.
		\item $ \lim_{n \to \infty} c\{ a_n \}=cs$ y $ \lim_{n \to \infty} (c+\{ a_n \}) =c+a$; para cualquier número $c$.
		\item $ \lim_{n \to \infty} \{ a_n \}\{ b_n \}= ab$.
		\item $ \lim_{n \to \infty} \dfrac{1}{\{ a_n \}} = 1/s$.
	\end{itemize}
\end{teo}

Un criterio muy útil para determinar convergencia de una sucesión es el siguiente.

\begin{defi}
	Una sucesión $\{ a_n \}$ es \Keyword{monótona} si se cumple que
	\[
		\{ a_n \}\leq\{ a_{n+1} \}\quad\text{o}\quad\{ a_n \}\geq\{ a_{n+1} \}.	
	\]
	El el primer caso diremos que es \emph{monótona creciente} y en el segundo caso \emph{monótona decreciente}.
\end{defi}

\begin{teo}\label{teo_monotono_acotado}
	Toda sucesión \emph{monótona y acotada} es convergente.
\end{teo}

\subsection{Subsucesiones}%
\label{sec:Subsucesiones}

\begin{defi}
	Dada una sucesión $\{ a_n \}$ consideremos una sucesión de números naturales $\{ n_k \}$ tal que $n_1<n_{2}<\dots$
	Entonces la sucesión $\{ a_{n_k} \}$ es una \Keyword{subsucesión} de $\{ a_n \}$.
\end{defi}

Si la sucesión $\{ a_n \}$ converge a $L$ entonces toda subsucesión de $\{ a_n \}$ también converge a $L$.

\begin{teo}
	\normalmarginpar\marginnote{Teorema de Bolzano-Weierstra{\ss}}
	Toda sucesión real acotada contiene una subsucesión convergente.a
\end{teo}

\begin{proof}
	Basta demostrar, por el teorema~\ref{teo_monotono_acotado}, que toda sucesión real contiene una subsucesión \emph{monótona}.
	Para esto se usa la noción de \emph{picos}\footnote{Véase el capítulo de sucesiones en~\cite{spivak_calculus_2008}}.
\end{proof}

\section{Sucesiones de Cauchy}%
\label{sec:Sucesiones de Cauchy}

\begin{defi}
	Definición a
\end{defi}
