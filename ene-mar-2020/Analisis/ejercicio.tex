\documentclass[fleqn,leqno,bigmar,draft]{tareas}

\title{Límites y Continuidad}
\author{Jhonny Lanzuisi, 15\,10759}
\Asignatura{Análisis 1}

\begin{document}
\maketitle
\marginnote{
	\begin{abstract}
		Ejercicio de \AsignaturaN: Límites y continuidad.
	\end{abstract}
	\tableofcontents
}

\section[Enunciado]{Enunciado\footnotemark}
\footnotetext{En \cite{spivak_calculus_2008}, Capítulo 6, ejercicio 17(d).}
Sea $f$ una función cuyas discontinuidades son \emph{todas evitables}. Es decir,
para todo $x$ el $ \lim_{y\to x}f(y) $ existe pero es distinto de $f(x)$. 
Sea
\[
	g(x) = \lim_{y\to x} f(y),
\]
demuestre que $g$ es continua.

\subsection{Solución}
Sea $a$ un punto en el dominio de $g$. Entonces, para todo $x,y\in\mathbb{R}$,
\begin{align*}
	\left\lvert x-a \right\rvert &= \left\lvert x-a+y-y \right\rvert\\
				     &= \left\lvert (y-a) + (x-y) \right\rvert\\
				     &\leq \left\lvert y-a \right\rvert + \left\lvert x-y \right\rvert\\
				     &= \left\lvert y-a \right\rvert + \left\lvert y-x \right\rvert.\tag{1}
\end{align*}

Como $f$ es continua en $a$ y en $x$, dado $\epsilon>0$ existen constantes
$\delta,\delta'$ (que dependen de $\epsilon$) tales que
\[
	(1)\leq\delta+\delta'
\]
implica
\begin{align*}
	\epsilon>\left\lvert f(y)- \lim_{y\to a} f(y) \right\rvert + \left\lvert f(y)- \lim_{y\to x} f(x) \right\rvert.\tag{2}
\end{align*}

Pero 
\begin{align*}
	(2)&=\left\lvert f(y)- \lim_{y\to a} f(y) \right\rvert + \left\lvert \lim_{y\to x} f(x) - f(y)\right\rvert\\
	   &>\left\lvert \cancel{f(y)}- \lim_{y\to a} f(y) + \lim_{y\to x} f(x) - \cancel{f(y)}\right\rvert\\
	   &= \left\lvert  \lim_{y\to x} f(y) - \lim_{y\to a} f(y)\right\rvert\\
	   &= \left\lvert g(x)-g(a) \right\rvert
\end{align*}

Y por lo tanto, siempre que
\[
	\left\lvert x-a \right\rvert<\delta+\delta'
\]
se tendrá que
\[
	\left\lvert g(x)-g(a) \right\rvert < \epsilon.
\]
Y se tienen entonces que $ \lim_{x\to a} g(x) = g(a)$, por lo que $g$ es continua en todo su dominio (pues $a$ se eligió como un elemento arbitrario del dominio de $g$).
\printbibliography[
heading=bibintoc,
title={Referencias}
]
\end{document}
