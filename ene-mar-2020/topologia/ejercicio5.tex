\documentclass[fleqn,leqno,letterpaper,11pt,final]{article}
% Misc TeX Settings
\hfuzz1pc
\overfullrule=2cm

% Language Setup
\usepackage{polyglossia}
\setmainlanguage[spanishoperators=all,]{spanish}
\PolyglossiaSetup{spanish}{,indentfirst=false}
\usepackage{csquotes} 

% Page dimensions
\usepackage[
	includehead,
	includefoot,
	top=1.5cm,
	bottom=1.5cm,
	left=2.5cm,
	right=8.5cm,
	marginparsep=0.5cm,
	marginparwidth=7cm,
]{geometry}

% Math
\usepackage{mathtools} 
\DeclareMathOperator{\Rea}{Re}
\DeclareMathOperator{\Ima}{Im}
\DeclareMathOperator{\car}{car}
\DeclareMathOperator{\traz}{tr}
\DeclareMathOperator{\gen}{gen}
\DeclareMathOperator{\mcm}{mcm}
\DeclareMathOperator{\id}{id}

% Text Fonts
\usepackage{unicode-math} 
\defaultfontfeatures{SmallCapsFeatures={LetterSpace=10},Scale=MatchLowercase,Numbers={OldStyle,Proportional}}
\setmainfont[
SmallCapsFont=* Caps,
]{Latin Modern Roman}
\setsansfont{Libertinus Sans}
\setmonofont[BoldFont=* Semibold]{Source Code Pro}
\newcommand{\dispfont}{} 
\frenchspacing
\setlength{\parindent}{1em}
\linespread{1.02}

% Microtype
\usepackage[
final,
% tracking=smallcaps,
% expansion=alltext,
% protrusion=true
]{microtype}
% \SetTracking{encoding=*,shape=sc}{50}

% Misc Packages
\PassOptionsToPackage{final}{graphicx}
\usepackage{%
	xcolor,%
	graphicx,%
	cancel,%
	booktabs,
	hyphenat,
	authoraftertitle,
	pdfpages
}

% Bibliography
\usepackage[
	backend=biber,
	backref=true,
	style=trad-abbrv,
	sorting=ynt
]{biblatex}
\addbibresource{C:/Users/Jhonny/Google-Drive/LaTeX/bib/general.bib}

% References
\usepackage{url} 
\usepackage{hyperref} 
\hypersetup{colorlinks=true,linkcolor=black,urlcolor=black}
\usepackage[spanish,nameinlink]{cleveref} 

% List Settings
\usepackage{enumitem} 
\setlist[enumerate]{label=\arabic*,left=-11pt}
\setlist[description]{font=\normalfont,leftmargin=\parindent}
\setlist[itemize]{label={\small\textbullet},left=-11pt}

% Margins: notes, figures, etc
\newcommand{\concretefont}{\sffamily} 
\usepackage{marginnote,sidenotes} 
\renewcommand*{\marginfont}{\small\concretefont}
\let\oldmarginpar\marginpar
\renewcommand{\marginpar}[1]{
	\oldmarginpar{\raggedright\small\concretefont #1}
}
\newcounter{nota}
\newcommand{\nota}[1]{\refstepcounter{nota}\textsuperscript{\thenota}%
	\marginpar{%
		\parbox{3.5cm}{\raggedright\thenota. #1}%
	}
}

% Custom \maketitle
\newcommand{\Asignatura}{}
\newcommand{\asignatura}[1]{\renewcommand{\Asignatura}{#1}}
% \setlength{\textparlen}{}
% \addtolength{\textparlen}{20pt}
\makeatletter
\def\@maketitle{%
  \newpage
  \null
  \let \footnote \thanks
  \begin{flushleft}\dispfont
	% \begin{minipage}{.25\textwidth}
	% \raggedleft\Asignatura
	% \end{minipage}
	% \enspace
	% \newsavebox{\tmpbox}\savebox{\tmpbox}{\parbox{.4\textwidth}{\raggedright\dispfont\@title}}%
	% \rule[-.9\ht\tmpbox]{1pt}{2\ht\tmpbox}\enspace%
	% \usebox{\tmpbox}%
	  {\addfontfeatures{LetterSpace=10}\MakeUppercase{\@title}\marginnote{\parbox{3cm}{\raggedright\@author}}\par\smallskip\titlerule}
	\medskip
	  {\small\raggedleft\Asignatura\ (0\the\month-\the\year)\par}
  	% \vskip 1em
  	% {}\par
	% {}\par
  \end{flushleft}
  % \marginpar{\@author\medskip}
  \vskip 1\baselineskip
}
\makeatother

% Caption setup
\usepackage{caption} 
\captionsetup{font={rm},justification=raggedright,singlelinecheck=false,skip=3pt}

% Section and subsection format
\usepackage[explicit]{titlesec}
\titleformat{\section}[hang]
{\raggedright\large\itshape}
	{\thesection}
	{1em}
	{#1}
	[]
\titlespacing*{\section}
	{0em}
	{1.5\baselineskip}
	{1.5\baselineskip}
\titleformat{\subsection}
{\flushleft\itshape}
	{\thesubsection}
	{.5em}
	{#1}
	[]
\titlespacing*{\subsection}
	{0em}
	{1\baselineskip}
	{1\baselineskip}
\titleformat{\paragraph}[runin]
	{\addfontfeatures{LetterSpace=10}\scshape}
	{}
	{0em}
	{#1}
	[.]
\titlespacing*{\paragraph}
	{0em}
	{0\parskip}
	{1\parskip}

% Table of contents setup
% \addto\captionsspanish{
% 	\renewcommand{\contentsname}{\concretefont Contenido}
% }
\let\oldtoc\tableofcontents
\renewcommand{\tableofcontents}{\marginpar{\bigskip\oldtoc}}
\usepackage{titletoc}
\titlecontents{section}
	[0em]
	{\vspace{-10pt}}
	{\contentsmargin{0pt}}
	{\contentsmargin{0pt}}
	{\contentspage}
	[\vspace{15pt}]
\titlecontents{subsection}
	[.5em]                              
	{\vspace{-11pt}}
	{\contentsmargin{0pt}\footnotesize}
	{\contentsmargin{0pt}\footnotesize}        
	{\contentspage}                 
	[\vspace{14pt}]

% Abstract redefine
	\renewenvironment{abstract}{
		\smallskip
		{\raggedright\large Resumen}\par\smallskip
	}{\par\medskip}
% Header and footer setup
\usepackage{fancyhdr}
\renewcommand{\headrulewidth}{0pt}
\setlength{\headheight}{14pt}
\pagestyle{fancy}
% \renewcommand{\sectionmark}[1]{\markright{#1}}
\fancyhf{}
\fancyhead[L]{\ifodd\value{page}\MyTitle\else\Asignatura\fi}
	\fancyhead[R]{\thepage}
\fancypagestyle{plain}{%
	\fancyhead[R]{}
	\fancyhead[L]{}
	\fancyfoot[R]{}%
	\fancyfoot[L]{}
	\fancyfoot[C]{}
}

% Theorem environments
\usepackage[thmmarks]{ntheorem}
	\theoremstyle{plain}
	\theoremindent0cm
	\theorempreskip{1\parskip}
	\theoremheaderfont{\normalfont}
	\theorembodyfont{\normalfont}
	\theoremseparator{.}
	\newtheorem{defi}{Definición}[section]
	\newtheorem{teo}{Teorema}[section]
	\newtheorem{cor}{Corolario}[teo]
	\newtheorem{prop}{Proposición}[section]
	\newtheorem{lem}{Lema}[section]
	\newtheorem{ejem}{Ejemplo}[section]
	\newtheorem{cejem}{Contraejemplo}[section]
	\newtheorem{ejer}{Ejercicio}[section]
	\theoremstyle{nonumberplain}
	\newtheorem{sol}{Solución}
	\newtheorem{obs}{Observación}
	\newtheorem{proof}{Demostración}
\renewcommand{\footnote}{\nota}
% \author{Jhonny Lanzuisi}


\title{Espacios conexos}
\author{Jhonny Lanzuisi, 15\,10759\\\url{jalb97@gmail.com}}
\asignatura{Topología 1}

\begin{document}
\maketitle
%\begin{abstract}
%Ejercicio del curso \Asignatura: espacios conexos. Imagina que es un resumen muy larg, como prueba de que an mal se veria en un documento como prueba de algo si como prueba
%\end{abstract}
\tableofcontents	
	

\section[Enunciado]{Enunciado}

\hspace{-.6em}\footnotemark Sea $X_{\alpha\in J}$ una familia indexada de espacios conexos. Seaa
\[
	X=\prod_{\alpha\in J} X_\alpha,
\]
y sea $a=a_\alpha$ un punto fijo de $X$. $\mathcal{V}_t$
\footnotetext{En \cite{munkres_topology_2014}: \S23, Ejercicio 10}
\begin{enumerate}
	\item Dado un subcojunto $K$ de $J$ finito, llamemos $X_K$ al subespacio de $X$ dado por los puntos
		$x$ tales que $x_\alpha=a_\alpha$ si $\alpha\not\in K$. Demuestre que $X_K$ es conexo.
	\item Demuestre que la unión $Y$ de todos los $X_K$ es conexa.
	\item Demuestre que $Y$ es la clausura de $X$ y concluya que $X$ es conexo.
\end{enumerate}

\subsection{Solución}
\paragraph{Primera Parte}
Sean $\alpha_1,\dots,\alpha_n$ los elementos de $K$, llamemos $P$ al producto $\prod_{\alpha\in K}X_\alpha$,
y consideremos la función 
$f:P\to X_K$ dada, para $x\in P$, por $f(x)=z$ donde
\[
	z_\alpha= \begin{cases}
		x_\alpha &\text{si}\; \alpha\in K,\\
		a_\alpha &\text{si}\; \alpha\in J-K.
	\end{cases}
\]
Entonces $P$ y $X_K$ son homeomorfos a traves de esta $f$. Como $P$ es conexo por
el teorema~\ref{conex:prod}, se tiene que $X_K$ es conexo por el teorema~\ref{conex:contfunct}.

\paragraph{segunda parte}
Sea $Y$ la union de los $X_K$. Como los elementos de cualquier $X_K$ coninciden con
$a$ en las coordenadas que pertenecen a $K$, se sigue que el punto $a$ esta en
todos los $X_K$ de forma casi inmediata. La conexidad de $Y$ es entonces consecuencia del
teorema~\ref{conex:union}.

\paragraph{tercera parte}
Elijamos un punto $(x_\alpha)\in X$. Queremos ver que todo entorno de $(x_\alpha)$ contiene
un punto de $Y$. Elijamos un entorno $U$ de $(x_\alpha)$. Como estamos bajo
la topología producto, este entorno $U$ es un producto de abiertos donde todos
menos una cantidad finita de ellos son iguales a los $X_\alpha$. Sea
$K=\{\alpha_1,\dots,\alpha_n\}$ esta cantidad finita de índices.

Entonces podemos hayar un punto $(y_\alpha)$ que esta en $X_K$ y  en $U$ de la siguiente manera:
hagamos $y_\alpha=a_\alpha$ para $\alpha\notin K$ y para los $\alpha\in K$ tomamos $y_\alpha$
tales que pertenecen a $U_\alpha$, donde los $U_\alpha$ son los términos
del producto, que conforma a $U$, que difieren de los $X_\alpha$.

Luego como $(y_\alpha)\in X_K$ entonces también pertenece a $Y$ y se tiene
que $X$ es la clausura de $Y$.

Por lo que $X$ es conexo por el teorema~\ref{conex:cerrad}. Todo lo anterior en realidad ha demostrado
el siguiente teorema.
\begin{teo}
	En la topología producto, el producto arbitrario de espacios conexos es conexo.	
\end{teo}
%\section{prueba 1}
%\section{prueba 2}%
%\section{prueba 3}%
%\section{prueba 4}%
%\section{prueba 5}
%\section{prueba 6}%
%\section{prueba 7}%
%\section{prueba 8}%


\section{Resultados Utilizados}
\begin{teo}\label{conex:contfunct}
	La imagen de un espacio conexo bajo una función continua es conexo
\end{teo}
\begin{teo}\label{conex:prod}
	El producto finito de espacios conexos es conexo.
\end{teo}
\begin{teo}\label{conex:union}
	La unión de una colección de subespacios conexos de un espacio $X$ que tienen un
	punto en común es conexa.
\end{teo}
\begin{teo}\label{conex:cerrad}
	Sea $A$ un subespacio conexo de $X$. Si $A\subset B\subset\bar A$, entonces $B$
	también es conexo.
\end{teo}
\begin{flushright}\scriptsize
	(Todos los resultados fueron tomados de \cite{munkres_topology_2014}, Cap 3)
\end{flushright}

\printbibliography[
heading=bibintoc,
title={Referencias}
]
\end{document}
