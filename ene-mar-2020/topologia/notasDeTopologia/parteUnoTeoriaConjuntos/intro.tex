El concepto central de este capítulo ---y pieza
 fundamental en la matemática moder\-na--- es, al menos
 en la superficie, tremendamente simple. Un
 \emph{conjunto} es un agregado de objetos, una
 colección o grupo de estos objetos. Así, tenemos que
 la colección de los estudiantes inscritos en la
 Universidad Simón Bolívar es un conjunto, como también
 lo es la cantidad de dígitos en la expansión decimal
 de $\pi$.

\begingroup Esta noción de conjunto, en principio
	simple e intuitiva, irá revelando su dificultad
	a medida que se resuelvan problemas y se avance
	un poco en los conceptos. 
\endgroup

	Los
	conjuntos son una construcción abstracta,
	pensada por una cabeza humana, que consiste en
	agrupar todos los objetos que cumplen con una
	cierta propiedad. Esta propiedad puede ser en
	principio cualquiera, aunque más adelante
	daremos formas precisas de enunciar las
	propiedades para no caer en ambigüedades.
	Entonces todos los números que tienen la
	propiedad de ser múltiplos de dos son un
	conjunto, como también lo es la colección de
	todos los hijos que son a la vez sus propios
	padres (este último conjunto, aparentemente
	contradictorio, es \emph{vacío}. La noción de
	vacío se verá mejor más adelante).

Es interesante notar que, si nos conformamos con la
definición que hemos dado hasta ahora y la tomamos como
definitiva, pueden surgir contradicciones e
inconsistencias. Quizás el ejemplo mas paradigmático es
el siguiente, dicho en la versión del mismo que el
autor de esta guía escuchó por primera vez.

\begin{ejem} Existe un pueblo, en
	una tierra muy lejana, donde trabaja un solo
	barbero. Pero este barbero tiene una exigencia
	peculiar a sus clientes: solo afeita a aquellos
	que no se afeitan a ellos mismos. Todo estaría
	bien con nuestro barbero si no se nos ocurriese
	la siguiente pregunta: ¿El barbero se afeita a
	si mismo?
	
	Veamos. Si el barbero se afeita a si mismo
	entonces, por la \emph{propiedad} especial que
	cumple nuestro barbero, se sigue que el barbero
	no se afeita a si mismo: una contradicción. De
	igual forma, si el barbero no se afeita a si
	mismo entonces el babero sería una persona que
	no se afeita a si misma y tendríamos, por la
	condición peculiar de nuestro barbero, que se
	afeita a si mismo: otra contradicción.
	
	Tenemos que, sin importar que respuesta demos a
	nuestra pregunta, siempre llegamos a una
	contradicción: una paradoja. Los sistemas que
	se comportan de esta forma se suelen llamar
	\emph{inconsistentes}.
	
	La paradoja de Russel puede formularse
formalmente, utilizando notación que no hemos explicado
aún, de la siguiente manera: Sea $R = \{ x \mid x
\notin x \}$ preguntémonos si $R\in R$. Se deja como un
ejercicio al lector volver después de la siguiente
sección y desarrollar la paradoja de Russel en lenguaje
formal.  
\end{ejem}

La lección que se saca de ejemplos como la paradoja de
Russel es que el conjunto nombrado no existe y que, en
general, ser capaz de nombrar un conjunto no es
condición suficiente para asegurar su existencia. Más
aún, no tenemos hasta ahora ninguna manera de definir
formalmente la noción de conjunto de tal forma que
contradicciones como las del ejemplo anterior no
ocurran. Por esta razón es que no intentaremos dar una
noción mas formal de la idea de conjunto, en cambio
daremos unos cuantos \emph{axiomas} que describen
\marginnote{Un \emph{axioma} es una verdad que asumiremos sin demostración. }
bastante bien como esperamos que se comporte un
conjunto. Y partiendo de estos axiomas construiremos el
resto de nuestra teoría.

