\chapter*{Ejercicios Resueltos}

\section{Espacios Topológicos}

\begin{ejer} Sea $ X $ un espacio topológico y $ A $ un
	subconjunto de $ X $.  Supongamos que para cada
	$ x\in A $ existe un conjunto abierto $ U $ que
	contiene a $ x $ tal que $ U\subset A $.
	Demuestre que $ A $ es abierto.  \end{ejer}

\begin{sol} Sea $ \{ U_i \} $ la familia de todos los
	conjuntos $ U $ asociados a cada elemento $x\in
	A$. Entonces, como $ U_i\subset A $ para todo $
	i $, el conjunto $ A $ puede escribirse como \[
	A=\bigcup U_i \] y dado que los $ U_i $ son
	cojuntos abiertos su unión ha de ser un
	conjunto abierto. Luego $ A $ es un conjunto
	abierto como se buscaba.  \end{sol}

\begin{ejer} Sea $ X $ un conjunto y $ \mathcal{T}_c$
	la colección de todos los subconjuntos $ U $ de
	$ X $ tales que $ X\setminus U $ es numerable o
	es todo $ X $. Demuestre que $ \mathcal{T}_c $
	es una topología sobre $ X $.  \end{ejer}

\begin{sol} Primero que todo, el conjunto vacío
	pertenece a $ \mathcal{T}_c $ debido a que $
	X\setminus\emptyset=X $ y $X\in \mathcal{T}_c$
	por definición. También se tiene que $ X $ es
	un conjunto abierto puesto que $ X\setminus
	X=\emptyset $ y el conjunto vacío es numerable
	por vacuidad.  \end{sol} 
